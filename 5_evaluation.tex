\section{Evaluation}
\label{sec:evaluation}

\begin{table}[t]
\begin{tabular}{l | l | l}
System & Version & Notes \\
\hline
Chapel & 1.18.0 & {\lstinline!--fast!} \\
Charm++ & 6.9.0 & {\lstinline!-optimize!} \\
Dask & 1.1.5 & \\
MPI(+X) & Cray MPICH 7.7.3 & {\lstinline!-O3!} \\
OmpSs & 17.12.1 & {\lstinline!-O3!} \\
OpenMP & Intel KMP 18.0.1.163 & {\lstinline!-O3!} \\
PaRSEC & Git {\lstinline!master!} branch & {\lstinline!-O3!} \\
Realm & Git {\lstinline!subgraph!} branch & {\lstinline!-O3!} \\
Regent & Git {\lstinline!subgraph!} branch & {\lstinline!-fflow-spmd 1!} \\
Spark & 2.3.0 & Scala 2.11.8, Java 8 \\
StarPU & 1.2.8 & {\lstinline!-O3!} \\
Swift/T & 1.4 & {\lstinline!-O3!} \\
TensorFlow & 1.12.0 & \\
X10 & Git {\lstinline!master!} branch & {\lstinline!-O3 -NO_CHECKS!}
\end{tabular}

\caption{System version and configuration notes.\label{tab:flags}}
\vspace{-0.5cm}
\end{table}


We present a comprehensive evaluation of Task Bench on up to 256
Haswell nodes of the Cori supercomputer~\cite{Cori}, a Cray XC40
machine. Cori Haswell nodes have 2 sockets with Intel Xeon E5-2698 v3
processors (a total of 32 physical cores per node), 128 GB RAM, and a
Cray Aries interconnect. We use GCC 7.3.0 for compiling all Task Bench
implementations, and (where applicable) the system default MPI
implementation, Cray MPICH 7.7.3. Additional versions and flags for the
various systems are shown in Table~\ref{tab:flags}.

\subsection{Peak Performance and Efficiency}
\label{subsec:peak-performance-and-efficiency}

\begin{figure}[t]
\centering
\includegraphics[width=\columnwidth]{figs/task-bench-results/compute/flops_stencil.pdf}
\vspace{-0.7cm}
\caption{FLOPS vs problem size (stencil, 1 node). Higher is better.\label{fig:flops}}
\vspace{-0.45cm}
\end{figure}

\begin{figure}[t]
\centering
\includegraphics[width=\columnwidth]{figs/task-bench-results/compute/efficiency_stencil.pdf}
\caption{Task granularity vs efficiency (stencil, 1 node). Lower is better.\label{fig:efficiency}}
\end{figure}


% Really want this to appear in the next section, but Latex forces it to be placed here to get the layout we want.
\begin{figure*}[t]

\subfloat[Stencil pattern.\label{fig:metg-compute-stencil}]{
\includegraphics[width=\columnwidth]{figs/task-bench-results/compute/metg_stencil.pdf}
}
\subfloat[Nearest pattern, 5 deps/task.\label{fig:metg-compute-nearest}]{
\includegraphics[width=\columnwidth]{figs/task-bench-results/compute/metg_nearest.pdf}
}

\subfloat[Spread pattern, 5 deps/task.\label{fig:metg-compute-spread}]{
\includegraphics[width=\columnwidth]{figs/task-bench-results/compute/metg_spread.pdf}
}
\subfloat[Nearest pattern, 5 deps/task, 4 independent graphs.\label{fig:metg-compute-4x-nearest}]{
\includegraphics[width=\columnwidth]{figs/task-bench-results/compute/metg_ngraphs_4_nearest.pdf}
}

\vspace{-0.15cm}
\caption{METG vs node count for different dependence patterns. Lower is better.\label{fig:metg-compute}}
\vspace{-0.25cm}
\end{figure*}


In theory, any system should be able to hit peak efficiency as long as
the kernels are well-written and of sufficiently large granularity. In
practice, achieving peak performance is tricky, subject to a number of
pitfalls, and unfortunately, frequently skipped in evaluations of
runtime system overhead. However, verifying that peak performance is
achieved is important to ensure that such evaluations of overhead and
efficiency are well-grounded and avoid common pitfalls in
configuration.

Figure~\ref{fig:flops} shows the FLOPS achieved with a compute-bound
kernel with varying problem sizes. This is the full version of
Figure~\ref{fig:flops-mpi}. In the best case, we measure peak FLOPS of
$1.26 \times 10^{12}$, which compares favorably with the officially
reported number of $1.2 \times 10^{12}$ \cite{Cori}. For the purposes
of measuring efficiency, we use our empirically determined number as
the baseline for 100\% efficiency.

The vast majority of systems achieve or nearly achieve peak FLOPS. Some
systems reserve a number of cores (usually 1 or 2) for internal
runtime usage; these systems take a minor hit in peak FLOPS compared
to systems which share all cores between the application and runtime. Some of the
slower systems struggle to achieve peak FLOPS, though in most cases
the curves suggest that performance would continue to improve if we
were to run larger problem sizes. Unfortunately, the excessive
computational cost of running such tests makes it prohibitively
expensive to run larger problem sizes. For example, the Spark job in
this configuration ran for over 6 hours.

Figure~\ref{fig:efficiency} shows efficiency (as a percentage of
peak FLOPS achieved) plotted against task granularity. This plot is
used to calculate METG(50\%), as described in Section~\ref{sec:metg}. The
red, dashed line shows the point where 50\% efficiency is achieved.

In most cases, task granularity has plateaued prior to the 50\% efficiency point,
though some systems continue to improve as efficiency continues
to drop. Accounting for this effect is one of the main arguments in
favor of METG with a reasonable efficiency threshold compared to
measuring task scheduling throughput in the presence of empty tasks
(effectively METG(0\%)). Empty tasks permit runtime
systems to make unrealistic tradeoffs which are not applicable to real
application workloads where some degree of efficiency is necessarily a
goal. Reporting numbers with empty tasks therefore gives unfair
advantage to systems that make such unrealistic tradeoffs.

\subsection{Baseline Overhead}

One of the basic questions when considering different programming
systems is how much overhead the system introduces. This can be a
tricky question to answer directly because some systems introduce
overhead \emph{inline} (i.e., by running system internal processes on
the same cores as application tasks), while other systems introduce
overhead \emph{out-of-line} (i.e., by dedicating one or more cores
solely to runtime use). Some systems, like Charm++, Realm, and Regent,
even support both configurations.

To answer this question, we use the METG metric to determine the
smallest task granularity which can be executed at a given level of
efficiency, as a proxy for overhead. Figure~\ref{fig:metg-compute}
shows how METG(50\%) varies with node count for a variety of
dependence patterns supported by Task Bench. The experiment is conducted
in a weak scale fashion, in which we keep the number of tasks per node
constant while increasing the number of nodes. The detailed explanations
of each dependence pattern are as follows: 

\begin{itemize}
\item Figure~\ref{fig:metg-compute-stencil} is a 1D stencil where each
  task depends on 3 other tasks (including the same point in the
  previous timestep).
\item Figure~\ref{fig:metg-compute-nearest} is a pattern where each
  task depends on 5 others, chosen to be as close as possible.
\item Figure~\ref{fig:metg-compute-spread} is a pattern where each
  task depends on 5 others, spread as widely as possible.
\item Figure~\ref{fig:metg-compute-4x-nearest} shows 4 identical
  copies of the nearest dependence pattern executing concurrently.
\end{itemize}

We observe in the results that the baseline overheads of different
systems can vary by over 5 orders of magnitude. It is worth
remembering when considering this metric that this is a \emph{minimum}
effective task granularity. Therefore applications with an average
task granularity of \emph{at least} this value can usually be expected
to execute efficiently. Typical average task granularity will
generally be determined by the application domain being
considered. Most notably, for data analytics workloads being run in
industrial data centers, the higher METG values observed for Spark and
TensorFlow are sufficient. In contrast, for high-performance
scientific simulations, task granularities in the millisecond range
are useful, as such applications communicate (e.g., for halo
exchanges) much more frequently.

\subsection{Scalability}
\label{subsec:scalability}

METG is a useful metric in part because it summarizes the basic task
scheduling overhead of a system in a single number. This makes it
possible to evaluate METG at different node counts (shown in
Figure~\ref{fig:metg-compute}) to see how it is impacted by changes in
communication topology and latency.

In general, the least complicated pattern (stencil) is most favorable
to MPI, as it represents a truly bulk synchronous task graph with no
opportunity for task parallelism. For the stencil pattern, the
dominating factor is the basic overhead of executing a task, which is
minimal for MPI as the Task Bench implementation is bulk synchronous
and simply executes tasks one after another in alternation with
communication phases. In contrast the asynchronous execution
mechanisms of the other systems are pure overhead in this scenario.

The gap between MPI and other systems shrinks as the complexity of the
communication pattern grows, and even reverses as task parallelism is
added in the form of multiple task graphs.

% FIXME: update this when we have more data

Most systems for high-performance computing are horizontally scalable,
but this is not true of all the systems included in this
evaluation. Lower is better in Figure~\ref{fig:metg-compute}, and a
flat line is ideal. Lines that rise with node count indicate less than
ideal scaling. Most notably, Spark is typically
intended for use in industrial data center applications with task
granularities measured in tens of seconds. Spark uses a centralized
controller, which limits throughput, and this is visible in the figure
as line for Spark immediately rises with node count. Keep in mind also
that Spark is being evaluated here with a non-trivial dependence
pattern which is relatively unrepresentative of Spark's normal use
cases. Spark is more efficient with trivial parallelism, as described
in Section~\ref{subsec:number-of-dependencies}. Systems like PaRSEC
and StarPU build the DAG on the fly, so they require to examine all tasks
of the entire DAG on each process. Even though DAG trimming is 
implemented on both systems to not launching tasks on a specific node
if data is not local, the overhead of sweeping all the tasks to 
check data localities goes up with the increasement of
node counts in the weak scale experiment, resulting in poor scalability. 
Even though the PTG implementation relieves the overhead of DAG expansion
from runtime to compile stage, the overhead of determining local tasks
based on data locality leads to un-horizontally scalable. 
The ``PaRSEC DTD Shard'' is an modified implementation of ``PaRSEC DTD''
by pre-partitioning the DAG when assuming data distribution is fixed 
(1D block is used in Task Bench). It removes the cost of sweeping the
entire DAG, and is therefore horizontally scalable.       

% FIXME: remove parsec dtd shard if no space. 

% FIXME: talk about overhead spanning 5 orders of magnitude

\subsection{Number of Dependencies}
\label{subsec:number-of-dependencies}

\begin{figure}[t]
\centering
\includegraphics[width=\columnwidth]{figs/task-bench-results/radix/metg_nearest.pdf}
\vspace{-0.6cm}
\caption{METG vs deps/task (nearest, 1 node). Lower is better.\label{fig:radix}}
\vspace{-0.45cm}
\end{figure}


The number of dependencies per task has a strong influence on the
overhead incurred by each system, as shown in
Figure~\ref{fig:radix}. This plot shows METG(50\%) for the nearest
dependence pattern, when varying the number of dependencies per task
from 0 to 9.

% FIXME: make sure these numbers are up to date for final paper

The ratio in METG between 0 and 3 dependencies per task ranges from
$1.3\times$ to $59\times$ (median $3.0\times$). From 1 to 3
dependencies, the ratio varies from $1.2\times$ to $15\times$ (median
$2.1\times$). The difference is most pronounced in systems which
perform runtime work inline. For example, MPI achieves an METG of 390
ns with 0 dependencies, but this rises to 4.6 us with 3 dependencies,
a factor of $12\times$ increase. This is unsurprising as in the case
of 0 dependencies, no \lstinline[language=C++]{MPI_Isend} calls are
issued at all, so MPI has nothing to do aside from executing kernels
as quickly as possible.

For this reason, we recommend evaluating the performance of
programming systems under realistic workloads, rather than with
trivial dependencies.

\subsection{Load Imbalance}

\begin{figure}[t]

\centering
\includegraphics[width=\columnwidth]{figs/task-bench-results/imbalance/efficiency_imbalance_2.0.pdf}

\vspace{-0.15cm}
\caption{Efficiency vs task granularity under load imbalance (nearest pattern, 5 dependencies per task, 4 task graphs, 1 node). Higher is better.\label{fig:efficiency-imbalance}}
\vspace{-0.1cm}
\end{figure}


One advantage of systems with asynchronous execution capabilities is
the ability to mitigate load imbalance, especially in the presence of
task parallelism. To quantify this effect,
Figure~\ref{fig:efficiency-imbalance} plots task granularity vs
efficiency curves under load imbalance with uniformly distributed task
durations. The task durations are generated with a deterministic
pseudo random number generator with a deterministic seed to ensure
that the task durations under all systems are identical.

The MPI implementation of Task Bench, being bulk synchronous,
suffers the most under load imbalance. The biggest
difference is at large task granularities, where the imbalance has the
effect of putting an upper bound on MPI efficiency. At smaller task
granularities the effect shrinks and may even reverse as systems hit
their fundamental limits due to overhead.

The difference between the remaining systems is due primarily to the
scheduling behavior of the various systems. For example, OpenMP
execute tasks via a thread pool, resulting in a form of work
stealing. This on-node work stealing, in combination with the task
parallelism due to the presence of 4 task graphs, is sufficient to
completely mitigate load imbalance. Systems that exploit task
parallelism but not work stealing on a node achieve lower peak
efficiency, but may achieve smaller task granularities as a result of
not relying on a shared global queue for tasks.

% FIXME: I am not sure if OpenMP has only 1 thread pool or multiple pools. 
% Systems like StarPU and PaRSEC have multiple pools, but still allows on-node
% work stealing. So I also think the last sentence " may achieve smaller 
% task granularities as a result of not relying on a 
% shared global queue for tasks." is not accurate. 

\subsection{Communication Hiding}

\begin{figure*}[t!]

\subfloat[16 bytes per task dependency.\label{fig:efficiency-communication-16}]{
\includegraphics[width=\columnwidth]{figs/task-bench-results/communication/efficiency_nodes_64_comm_16.pdf}
}
\subfloat[256 bytes per task dependency.\label{fig:efficiency-communication-256}]{
\includegraphics[width=\columnwidth]{figs/task-bench-results/communication/efficiency_nodes_64_comm_256.pdf}
}

\subfloat[4096 bytes per task dependency.\label{fig:efficiency-communication-16}]{
\includegraphics[width=\columnwidth]{figs/task-bench-results/communication/efficiency_nodes_64_comm_4096.pdf}
}
\subfloat[65536 bytes per task dependency.\label{fig:efficiency-communication-256}]{
\includegraphics[width=\columnwidth]{figs/task-bench-results/communication/efficiency_nodes_64_comm_65536.pdf}
}

\caption{Task granularity vs efficiency for varying communication (spread pattern, 5 dependencies per task, 4 task graphs, 64 nodes). Lower is better.\label{fig:efficiency-communication}}
\end{figure*}


Also of interest is the ability of asynchronous systems to
hide communication latency in the presence of task
parallelism. Figure~\ref{fig:efficiency-communication} plots task
granularity vs efficiency curves for varying amounts of
communication. The degree of communication is configured by the number
of bytes produced by each task (and therefore communicated with each
task dependency).

Asynchronous systems such as Charm++ are able to show two benefits in
these plots. First, by overlapping communication with computation,
such systems are able to execute smaller task granularities at higher
levels of efficiency compared to the bulk synchronous MPI
implementation. Second, the asynchrony also makes the curves smoother,
as spikes in latency due to interference from other jobs can be
mitigated, leading to more predictable performance, especially at
smaller message sizes.

% FIXME why MPI can be affected by other jobs but asynchronous systems can not?

