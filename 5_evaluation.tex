\section{Evaluation}
\label{sec:evaluation}

\begin{table}[t]
\begin{tabular}{l | l | l}
System & Version & Notes \\
\hline
Chapel & 1.18.0 & {\lstinline!--fast!} \\
Charm++ & 6.9.0 & {\lstinline!-optimize!} \\
MPI & Cray MPICH 7.7.3 & {\lstinline!-O3!} \\
OmpSs & 17.12.1 & {\lstinline!-O3!} \\
OpenMP & Intel KMP 18.0.1.163 & {\lstinline!-O3!} \\
PaRSEC & Git {\lstinline!master!} branch & {\lstinline!-O3!} \\
Realm & Git {\lstinline!subgraph!} branch & {\lstinline!-O3!} \\
Regent & Git {\lstinline!subgraph!} branch & {\lstinline!-fflow-spmd 1!} \\
Spark & 2.3.0 & Scala 2.11.8, Java 8 \\
StarPU & 1.2.4 & {\lstinline!-O3!} \\
Swift/T & 1.4 & {\lstinline!-O3!} \\
TensorFlow & 1.12.0 & \\
X10 & Git {\lstinline!master!} branch & {\lstinline!-O3 -NO_CHECKS!}
\end{tabular}
\end{table}


We now present a comprehensive evaluation of Task Bench on up to 256
Haswell nodes of the Cori supercomputer~\cite{Cori}, a Cray XC40
machine. Cori Haswell nodes have 2 sockets with Intel Xeon E5-2698 v3
processors (total of 32 physical cores per node), 128 GB RAM, and a
Cray Aries interconnect. We use GCC 7.3.0 for compiling all Task Bench
implementations, and (where applicable) the system default MPI
implementation, Cray MPICH 7.7.3. Additional versions and flags for the
various systems are shown in Table~\ref{tab:flags}.

\subsection{Peak Performance and Efficiency}

\begin{figure}[t]
\centering
\includegraphics[width=\columnwidth]{figs/task-bench-results/compute/flops_stencil.pdf}
\caption{FLOPS vs problem size (stencil, 1 node). Higher is better.\label{fig:flops}}
\end{figure}

\begin{figure}[t]
\centering
\includegraphics[width=\columnwidth]{figs/task-bench-results/compute/efficiency_stencil.pdf}
\caption{Task granularity vs efficiency (stencil, 1 node). Lower is better.\label{fig:efficiency}}
\end{figure}


In theory, any system should be able to hit peak efficiency as long as
the kernels are well-written and of sufficiently large granularity. In
practice, achieving peak performance is tricky, subject to a number of
pitfalls, and unfortunately, frequently skipped in evaluations of
runtime system overhead. However, verifying that peak performance is
achieved is important to ensuring that such evaluation of overhead and
efficiency are well-grounded and avoid common mistakes in
configuration.

Figure~\ref{fig:flops} shows the FLOPS achieved with a compute-bound
kernel with varying problem sizes. This is the full version of
Figure~\ref{fig:flops-mpi}. In the best case, we measure peak FLOPS of
$1.26 \times 10^{12}$, which compares favorably with the officially
reported number of $1.2 \times 10^{12}$ \cite{Cori}. For the purposes
of measuring efficiency we use our emperically determined number as
the baseline for 100\% efficiency.

The vast majority of systems achieve at or near peak FLOPS. A number
of systems reserve a number of cores (usually 1 or 2) for internal
runtime usage; these systems take a minor hit in peak FLOPS compared
to systems which use all cores for application tasks. Some of the
slower systems struggle to achieve peak FLOPs, though in most cases
the curves suggest that performance would continue to improve if we
were to run larger problem sizes. Unfortunately, the excessive
computational cost of running such tests makes it prohibitively
expensive to run larger problem sizes. For example, the Spark job in
this configuration ran for over 5 hours.

Figure~\ref{fig:efficiency} shows the efficiency (as a percentage of
peak FLOPS achieved) plotted against task granularity. This plot is
used to calculate METG, as described in Section~\ref{sec:metg}. The
red, dashed line shows the point where 50\% efficiency is achieved.

In most cases, task granularity has plateaued prior to this point,
though a number of systems continue to improve as efficiency continues
to drop. Accounting for this effect is one of the main arguments in
favor of METG with a reasonable efficiency threshold compared to
measuring task scheduling throughput in the presence of empty tasks
(effectively the equivalent of METG(0\%)). Empty tasks permit runtime
systems to make unrealistic tradeoffs which are not applicable to real
application workloads where some degree of efficiency is necessarily a
goal. Reporting numbers with empty tasks therefore gives unfair
advantage to systems that make such unrealistic tradeoffs.

\subsection{Overhead and Scalability}

\begin{figure*}[t]

\subfloat[Stencil.\label{fig:metg-compute-stencil}]{
\includegraphics[width=\columnwidth]{figs/task-bench-results/compute/metg_stencil.pdf}
}
\subfloat[Nearest 5 neighbors.\label{fig:metg-compute-nearest}]{
\includegraphics[width=\columnwidth]{figs/task-bench-results/compute/metg_nearest.pdf}
}

\subfloat[Spread 5 neighbors.\label{fig:metg-compute-spread}]{
\includegraphics[width=\columnwidth]{figs/task-bench-results/compute/metg_spread.pdf}
}
\subfloat[Nearest 5 neighbors, 4 simultaneous task graphs.\label{fig:metg-compute-4x-nearest}]{
\includegraphics[width=\columnwidth]{figs/task-bench-results/compute/metg_ngraphs_4_nearest.pdf}
}

\caption{METG vs node count for different dependence patterns.\label{fig:metg-compute}}
\end{figure*}


\subsection{Number of Dependencies}

\subsection{Load Imbalance}

\subsection{Communication Hiding}
