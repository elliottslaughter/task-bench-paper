\documentclass[sigconf,review]{acmart}

\usepackage{listings}
\usepackage{subfig}
\usepackage{grffile} % fixes graphicx to handle file names with dots

% fix footnotes in tables
\usepackage{footnote}
\makesavenoteenv{table}

% for multi-line comments
\newcommand{\zap}[1]{}

\lstdefinelanguage{Chapel}
  {morekeywords={coforall,forall},
  sensitive=true,
  morecomment=[l]{//},
  morecomment=[s]{/*}{*/},
  morestring=[b]",
  morestring=[b]',
}

\lstdefinelanguage{X10}
  {morekeywords={},
  sensitive=true,
  morecomment=[l]{//},
  morecomment=[s]{/*}{*/},
  morestring=[b]",
  morestring=[b]',
}

\lstset{
  captionpos=b,
  basicstyle=\ttfamily\small\linespread{0.8},
  numbers=left,
  numberstyle=\tiny,
  stepnumber=1,
  keepspaces=true,
  showlines=true,
  belowskip=-5pt,
  numbersep=3pt,
}

\copyrightyear{2019}
\acmYear{2019}
\setcopyright{acmlicensed}
\acmConference{SC19}{Nov. 2019}{Denver, CO, USA}
\acmPrice{15.00}
\acmDOI{10.1145/1122445.1122456}
\acmISBN{978-1-4503-9999-9/18/06}

\begin{document}

\title{Task Bench: A Parameterized Benchmark for Evaluating Parallel Runtime Performance}

% Note: double-blind submission, add authors later
%% \author{Name}
%% \affiliation{%
%%   \institution{Institution}}
%% \email{Email}

% 149 words

\begin{abstract}
We present Task Bench, a \emph{parameterized} benchmark designed to
explore the performance of parallel and distributed
programming systems under a variety of application scenarios. Task
Bench dramatically lowers the barrier to benchmarking and comparing multiple programming systems by making
the implementation for a given system orthogonal to the benchmarks
themselves: every benchmark constructed with Task Bench
runs on every Task Bench implementation. Furthermore,
Task Bench's parameterization enables a wide variety of benchmark
scenarios that distill the key characteristics of larger applications.

We conduct a comprehensive study with
15 programming systems on up to 256 Haswell nodes of the Cori
supercomputer. We introduce a novel metric, \emph{minimum effective task
  granularity} to study the baseline runtime overhead of each
system. Running at scale, 100\textmu{}s-long tasks are the finest
granularity that any system runs efficiently with current
technologies. We also study each system's
scalability, ability to hide communication and mitigate load
imbalance.
\end{abstract}


% http://dl.acm.org/ccs.cfm
\begin{CCSXML}
<ccs2012>
<concept>
<concept_id>10010147.10010169</concept_id>
<concept_desc>Computing methodologies~Parallel computing methodologies</concept_desc>
<concept_significance>500</concept_significance>
</concept>
<concept>
<concept_id>10010147.10010919</concept_id>
<concept_desc>Computing methodologies~Distributed computing methodologies</concept_desc>
<concept_significance>500</concept_significance>
</concept>
</ccs2012>
\end{CCSXML}

\ccsdesc[500]{Computing methodologies~Parallel computing methodologies}
\ccsdesc[500]{Computing methodologies~Distributed computing methodologies}

\keywords{Task Bench, METG, task-based runtimes, performance evaluation}

\maketitle

\section{Introduction}
\label{sec:introduction}

The challenge of parallel and distributed computation has led to a
wide variety of proposals for programming models, languages, and
runtime systems. While these systems are well-represented in the literature, comprehensive and comparative performance evaluations
remain difficult to find. Our
goal in this paper is to develop a useful framework for
comparing the performance of parallel and distributed programming
systems, to help users and developers evaluate the performance tradeoffs of these systems.

Existing approaches to this problem focus on \emph{proxy-}/\emph{mini-apps}
or \emph{microbenchmarks}. These smaller codes distill key
computational characteristics of larger applications: mini-apps are
often derived from a larger code, and thus inherit some subset of its
properties, while benchmarks are typically chosen to reflect a more
narrow set of behavior(s). In either case, while a variety of insight
can be gained, the overall programming effort required is proportional to the
product of the number of systems and behaviors being
evaluated. Few published studies compare more than a
handful of systems~\cite{LULESH13, Deakin19}.

\zap{
One approach to comparing performance is through \emph{proxy-} or
\emph{mini-apps}. Because they distill key computational characteristics of larger
applications, mini-apps offer insight
without the expense of developing production codes. However, despite the name, our experience is that
mini-apps still require significant investment to develop
to the level of quality needed for useful benchmarking. In many cases,
the effort to tune for performance exceeds the effort to develop a correct implementation. As a result, implementations of mini-apps
often do not reach the level of maturity required to compare
systems. Few published studies compare more than a handful of systems~\cite{LULESH13, Deakin19}.
} % zap

We present Task Bench, a parameterized benchmark for exploring the performance
of parallel and distributed programming systems under a
variety of conditions.  The key property of Task Bench is that it completely separates
the system-specific implementation from the implementation
of the benchmarks themselves.
In all previous benchmarks we know of, the effort to implement $m$ benchmarks on $n$
systems is $\mathcal{O}(mn)$.  Task Bench's design reduces this work to $\mathcal{O}(m + n)$,
enabling dramatically more systems and benchmarks to be explored for the same amount of programming
effort.  New benchmarks created with Task Bench
immediately run on all systems, and new systems that implement the Task Bench interface immediately run all
benchmarks. 

Benchmarks in Task Bench are based on the observation that regardless
of the programming system in which an application is written, many
applications can be modeled as coarse-grain units of work, called
\emph{tasks}, with dependencies between tasks representing the
communication and synchronization required for parallel and
distributed execution. By explicitly modeling the \emph{task graph}
(with tasks as vertices and dependencies as edges), we make it
possible to explore a wide variety of patterns relevant to
parallel and distributed computing: trivial parallelism, halo exchanges (as
in structured and unstructured mesh codes), sweeps (as
in the discrete ordinates method of simulating radiation), FFTs, trees
(for divide and conquer), DNNs, graph analytics, etc. Tasks execute kernels with a
variety of computational properties, including compute- and
memory-bound loops of varying duration. Dependencies can be configured to carry communication payloads of varying size. Finally, multiple
(potentially heterogeneous) task graphs can be executed concurrently
to introduce task parallelism into the workload. Together, these
elements enable the exploration of a large space of application
behaviors---and make it easy to explore cases limited by runtime
overhead as well as ones where computation or communication is
dominant.

Adding a system to Task Bench involves implementing a set of standard
services, such as executing a task or data transfer. Though
benchmarks are described in terms of task graphs, this is simply a
convenient representation of the computation, and the underlying
system need not provide any native support for tasks. We provide
Task Bench implementations in systems as diverse as MPI and
Spark. Task Bench provides a core API that encapsulates functionality
shared among systems, which reduces implementation effort and makes it
much easier to achieve truly apples-to-apples comparisons between
systems.

This approach has allowed us to benchmark 15 very different parallel
and distributed programming systems (see
Table~\ref{tab:systems}).  By running all systems on common benchmarks
we were able to quantify phenomena
that have never before been measured.
Most strikingly, the overheads of systems we examine vary by more than five orders
of magnitude, with popular, widely used systems at both ends of the spectrum!  Clearly,
slower systems have ``good enough'' performance for some applications, while presumably
providing advantages in programmer productivity.

How does one predict whether performance will be good enough for a given application?
The most commonly reported measures,
\emph{weak} and \emph{strong} scaling, do not directly characterize
the performance of the underlying
programming system. Weak scaling can hide arbitrary amounts of runtime
system overhead by using sufficiently large problem sizes, and strong
scaling does not separate runtime system overhead from application costs
(such as communication) that scale with the number of nodes when
using progressively larger portions of a machine. 

To characterize the contribution of runtime overheads
to application performance, and as an example of the novel studies that can be done
with Task Bench, we introduce a new metric called
\emph{minimum effective task granularity} (METG). Intuitively, for a given
workload, METG(50\%) is the smallest task granularity that maintains
at least 50\% efficiency, meaning that the application achieves at
least 50\% of the highest performance (in FLOP/s, B/s, or other
application-specific measure) achieved on a given
machine. The efficiency bound in METG is a key innovation over
previous approaches, such as \emph{tasks per second} (TPS), that fail
to consider the amount of useful work performed (if tasks are
non-empty~\cite{Canary16, Armstrong14}) or to perform useful work at all (if tasks are empty~\cite{LegionTracing18}).

METG captures the important essence of a
weak or strong scaling study, the behavior at the limit of
scalability. For weak scaling, METG(50\%) corresponds to the
smallest problem size that can be weak-scaled with 50\%
efficiency. For strong scaling, METG(50\%) can be used to compute the
scale at which efficiency can be expected to dip below 50\%.
We note that METG(50\%) for a given runtime system will
vary with the application and the underlying hardware---i.e., METG(50\%)
is not a constant for a given system, but we find that systems have
a characteristic range of METG(50\%) values and that there is additional insight
in the reasons that METG can vary.

A lower METG does not necessarily mean that
performance for a particular workload is better. Two systems with METG(50\%) of 100~\textmu{}s and 1~ms,
respectively, running an application with 10~ms average task granularity, are both likely to perform well. Only when task
granularity approaches (or drops below) METG(50\%) will they
likely diverge. METG identifies the regime in which a
given system can deliver good performance, and explains how
different systems coexist with runtime overheads that vary by orders of magnitude.

%
% FIXME:  Move these two paragraphs out of the introduction and into one of the METG sections.
%
\zap{
Task Bench and METG address issues common in limit studies of runtime
systems for parallel and distributed programming. Such studies often
employ the metric \emph{tasks per second} (TPS), which is almost
universally measured with trivial (i.e., no) dependencies \cite{LegionTracing18, Canary16, Armstrong14}. While
phrased in terms of tasks, TPS can be measured for any system as long
as the application in question has identifiable units of work that run
to completion without interruption. TPS is an upper bound on
runtime-limited application throughput. But it is not a tight bound, as the
cost of nontrivial dependencies can be significant. This issue can be easily fixed by running nontrivial
configurations of Task Bench.

There is another, deeper issue with TPS. TPS may be measured with
empty tasks~\cite{LegionTracing18} or with tasks of some
duration~\cite{Canary16, Armstrong14}. When using empty tasks, the
resulting upper bound on task scheduling throughput fails to represent
useful work within a realistic application. With non-empty tasks,
\emph{where the efficiency of the overall application is not
 reported}, TPS is not a measurement of runtime-limited
performance. Large tasks may be used to hide any amount of runtime
overhead, while small tasks may result in a drop in total
application throughput even as TPS increases. Only by constraining
efficiency, as in METG, can we meaningfully measure how runtime
overhead impacts the ability to perform useful application work.
} % zap

% Seems to be nearly impossible to get rid of the hyphen, so just give
% up and allow this paragraph to break across it.
\let\oldbrokenpenalty\brokenpenalty
%% \brokenpenalty=0

We conduct a comprehensive study of all 15 Task Bench implementations on
up to 256 Haswell nodes of the Cori supercomputer~\cite{Cori}.
%% :
%% Chapel \cite{Chapel15}, Charm++ \cite{Charmpp93}, Dask \cite{Dask15}, MPI \cite{MPI}, MPI+X (OpenMP, CUDA),
%% OmpSs \cite{OmpSs11}, OpenMP \cite{OpenMPSpec40},
%% PaRSEC \cite{PARSEC13}, Realm \cite{Realm14}, Regent \cite{Regent15},
%% Spark \cite{Spark10}, StarPU \cite{StarPU11},
%% Swift/T \cite{Wozniak13}, TensorFlow \cite{TensorFlow15}, and
%% X10 \cite{X1005}.
Using METG, we find that a number of factors---node
count, accelerators, and complex dependencies, among
others---individually or in combination contribute to an order of
magnitude or greater increase in METG, even in systems with the lowest
overheads. While some systems can achieve sub-microsecond METG(50\%) in
best-case scenarios, we show that a more realistic
bound for running nearly any application at scale is 100~\textmu{}s with
current technologies. Our study includes several asynchronous systems
designed to provide benefits such as overlapped computation
and communication. While small-scale benchmarks of these systems
suffer from increased overhead, we find that the benefits of these systems become
tangible at scale (provided the runtime overhead doesn't increase
beyond about 100~\textmu{}s per task).

%% \brokenpenalty=\oldbrokenpenalty

% Performance issues:
% Chapel (communication performance, core pinning)
% Dask (asymptotic complexity of dask.delayed)
% PaRSEC (task pruning with nontrivial task graphs)
% Realm (DMA system overhead)
% TensorFlow (constant folding not parallel)

Beyond comparative study, the ability to explore a large configuration
space also enables the discovery of bugs in the underlying systems. We
found five performance issues, ranging from communication efficiency
(Chapel, Realm), to the efficiency of task pruning, analysis and
constant folding (PaRSEC, Dask and TensorFlow). Three have been fixed
and all have been acknowledged by the developers of the respective
systems. (All were either fixed or worked around in our experiments.)
In some cases these correspond to order of magnitude or even
asymptotic improvements in the performance of the underlying
systems---benefits which apply well beyond Task Bench to all classes
of applications. The bugs are described in more detail in Section~\ref{sec:case-study}.

The paper is organized as follows: Section~\ref{sec:task-bench}
describes the Task Bench design. Section~\ref{sec:implementation}
discusses implementations in 15 systems.  Section~\ref{sec:metg}
defines METG and its relationship to quantities of interest to
application developers.  Section~\ref{sec:evaluation} provides a
comprehensive evaluation on Cori. Section~\ref{sec:case-study} describes bugs found with Task Bench. Section~\ref{sec:related-work} relates to
previous efforts; Section~\ref{sec:conclusion} concludes.

\section{METG (50\%)}
\label{sec:metg}

\begin{figure}[t]
\centering
\includegraphics[width=\columnwidth]{figs/task-bench-results/compute/flops_stencil_mpi.pdf}
\vspace{-0.6cm}
\caption{MPI FLOP/s vs problem size (stencil, 1 node).\label{fig:flops-mpi}}
\vspace{-0.1cm}
\end{figure}

\begin{figure}[t]
\centering
\includegraphics[width=\columnwidth]{figs/task-bench-results/compute/efficiency_stencil_mpi.pdf}
\caption{MPI task granularity vs efficiency (stencil, 1 node).\label{fig:efficiency-mpi}}
\end{figure}


The \emph{minimum effective task granularity}, METG(50\%), for an application $A$ is
the smallest average task granularity (i.e., task duration) such that $A$
achieves overall efficiency exceeding 50\%. For example, in
a compute-bound application efficiency might be measured as the
percentage of the available FLOPS achieved. On a machine with a peak
capability of 1.26 TFLOPS per node, METG(50\%) would correspond to
the smallest task granularity achieved while maintaining at least 0.63
TFLOPS. Relative parallel efficiency (vs a sufficiently large problem
size) can also be used in cases where the application isn't easily
characterized by peak resource usage.

Figures~\ref{fig:flops-mpi} and \ref{fig:efficiency-mpi} show how METG is
calculated. The application, in this case an MPI implementation of the
Task Bench stencil pattern in Figure~\ref{fig:task-graphs-stencil}, is
run on a single Haswell node of Cori with a problem size large enough to
ensure that it achieves peak FLOPS. This starting point confirms
that the application is properly configured and that the choice of
efficiency metric is achievable. The problem
size is then repeatedly reduced while maintaining exactly the same hardware
configuration (in particular, the same number of nodes). The
expectation is that as problem size shrinks,
performance will begin to drop and eventually approach zero, as shown in Figure~\ref{fig:flops-mpi}. Systems
with lower runtime overheads maintain higher performance at smaller
problem sizes compared to systems with higher overheads.

To calculate METG, the data is replotted along axes of efficiency
(i.e., as a percentage of the peak FLOPS achieved) and task
granularity (i.e., $\text{wall time} \times \text{number of
  cores}/\text{number of tasks}$), as shown in Figure~\ref{fig:efficiency-mpi}. Note that a \emph{task} is defined
broadly to be any continuously-executing unit of application code,
and thus it makes sense to discuss tasks even in systems
without an explicit notion of tasking, such as in MPI programs that
are written in a bulk synchronous style. In this case, the tasks run a
compute-bound kernel included in the Task Bench implementation,
described in more detail in Section~\ref{sec:task-bench}.

In Figure~\ref{fig:efficiency-mpi}, we see that
efficiency starts near 100\%, and that initially task granularity
drops quickly with minimal change in efficiency. However as task
granularity shrinks further, efficiency begins to drop as well, leading
eventually to a plateau in task granularity. Intuitively, this plateau
represents the fundamental baseline overhead of the system, the point
below which the system simply cannot execute tasks efficiently.

METG(50\%) is the intersection of this curve with a vertical line through
Figure~\ref{fig:efficiency-mpi} at 50\% efficiency. In the figures, the red, dashed line shows the 50\% efficiency
level. At 50\% efficiency, MPI achieves an average task granularity of
4.6 us, thus the METG(50\%) of MPI is 4.6 us for this configuration of
the application. We use 50\% because it is a reasonable level of
efficiency in practice. Lower values would likely not be reasonable,
because operators of computing facilities prefer higher levels of
utilization. Values above 90\% can misrepresent the performance of
some systems (see
Section~\ref{subsec:peak-performance-and-efficiency}).

METG is an appealing metric in part because it has a well-defined
relationship with quantities of interest to application developers,
namely weak and strong scaling. Figures~\ref{fig:weak-scaling-mpi} and
\ref{fig:strong-scaling-mpi} show the behavior of the MPI Task Bench stencil's weak and strong
scaling, respectively, at a variety of problem sizes. Intuitively, at
larger problem sizes MPI is perfectly efficient. This can be seen at
the top of each figure, with flat lines when weak scaling and
ideally-sloped downward lines when strong scaling. Inefficiency begins
to show at smaller problem sizes, towards the bottom of the graph,
where lines become more compressed. At the
very bottom, the lines compress together as running time becomes dominated by overhead. The shape of the curve at the bottom is the same
for both weak and strong scaling; this is the shape of
the METG curve.

\begin{figure}[t]
\centering
\includegraphics[width=\columnwidth]{figs/task-bench-results/compute/weak_mpi.pdf}
\vspace{-0.6cm}
\caption{MPI weak scaling with problem size per node (stencil).\label{fig:weak-scaling-mpi}}
\vspace{-0.1cm}
\end{figure}

\begin{figure}[t]
\centering
\includegraphics[width=\columnwidth]{figs/task-bench-results/compute/strong_mpi.pdf}
\caption{MPI strong scaling (stencil).\label{fig:strong-scaling-mpi}}
\end{figure}


METG therefore has a direct relationship with the smallest problem
size that can be weak scaled to a given node count with a given level
of efficiency. With any smaller problem size, 
the runtime overhead begins to dominate useful
computation. Similarly, METG corresponds to the point at which strong
scaling can be expected to stop; as strong scaling runs start with a
larger problem size and progressively shrink it until overheads begin
to dwarf the gains due to continued scaling. In
Figures~\ref{fig:weak-scaling-mpi} and \ref{fig:strong-scaling-mpi},
the dashed, bright red line show the value of METG(50\%) at each node
count.

The METG metric has another useful property. Because METG is measured ``in place'' (i.e.,
without changing the number of nodes or cores available to the
application), METG isolates effects that are
due to shrinking problem size from effects that are due to
increased communication and other resource issues as
progressively larger portions of the machine are used.

Section~\ref{sec:evaluation} contains a comprehensive evaluation of
METG including how it changes with node count for a wide variety of
programming systems.

\section{Task Bench}
\label{sec:task-bench}

In order to measure properties such as METG for a wide variety of
systems and application scenarios, we developed a novel
\emph{parameterized} benchmark called Task Bench. Task Bench
implementations execute a \emph{task graph}, with tasks for each point
in an \emph{iteration space} and dependencies between tasks determined
by a \emph{dependence relation}. Each task can execute any one of a
number of kernels (compute-bound, memory-bound, etc.) and can generate
a configurable amount of result data in order to control the amount of
communication that must be performed with each task dependency. Thus Task Bench can be used to
rapidly explore a large space of possible application scenarios in
order to understand the performance behaviors of parallel and
distributed languages and runtime systems.

For simplicity but without loss of generality, the iteration space in
Task Bench is constrained to be 2-dimensional, with time flowing along
the vertical axis and parallel tasks along the
horizontal. Tasks can depend only on tasks from the immediately
preceding time step. Many problems can be described by a task graph of
constant width, in which case the iteration space is rectangular, but
Task Bench is capable of handling non-rectangular iteration spaces,
such as required by sweeps and tree-based dependence
patterns. Figure~\ref{fig:task-graphs} shows a number of sample task
graphs that can be implemented with Task Bench.

% FIXME: Make figure showing example task graphs

Dependencies between tasks are controlled by a configurable dependence
relation. Dependencies are conceived as flowing backwards in time. The
dependence relation enumerates for each task, the set of tasks in the
previous time step that task depends on. This permits a wide variety
of dependence patterns to be implemented that are relevant to real
applications in high-performance scientific computing: stencils,
sweeps, FFTs, trees, and so on. Dependence patterns may also be
parameterized, such as picking the $K$ nearest neighbors, or $K$
distant neighbors. These dependence patterns can be seen in
Figure~\ref{fig:task-graphs}.

Despite its generality, Task Bench is easy to implement, making it
tractable to develop a suite of high-quality implementations. Most
Task Bench implementations are several hundred lines long, with the
shortest being 88 lines and the longest being 1500 lines of executable
code. Much of this is possible because the core aspects of generating
task graphs and enumerating dependencies are encapsulated in a core
library that is shared among all the Task Bench implementations. The
core library also includes implementations of the kernels, ensuring
that the kernels are identical in all cases, eliminating a potential
source of performance disparity which can be a pitfall for mini-apps
implementations. Finally, the Task Bench core library manages the
parsing of input parameters and the display out output results,
ensuring that all implementations behave in a uniform manner, and can
be scripted consistently.

The task implementation in the Task Bench core library is fully
validating. Because the task graph configuration is explicitly
represented (though unmaterialized) in Task Bench, this representation
can be queried to determine exactly what dependencies a task should
expect. The output of every task in Task Bench is uniquely generated,
and all inputs are verified. An assertion is thrown if validation
fails. This ensures that every execution of Task Bench, if it
completes successfully, is correct. This validation provides high
assurance that all Task Bench implementations are correct, and
safeguards against corner cases that may only be exercised at
scale. An evaluation of the performance impact of
this validation showed it to be less than 3\% at the smallest task
granularities in any Task Bench implementation, with a negligible
effect on overall results.

\section{Implementations}
\label{sec:implementation}

\begin{table}[t]
\begin{tabular}{l | l | l | l | l}
System & Paradigm & Parallelism & Distrib. & Network \\
\hline
Chapel & multi-resolution & \emph{expl.}, impl. & yes & uGNI\footnote{Chapel uses GASNet to support non-Cray networks.} \\
Charm++ & actor model & explicit & yes & uGNI\footnote{Charm++ provides additional backends for other networks.} \\
Dask & task-based & implicit & yes & sockets \\
MPI & message passing & explicit & yes & uGNI\footnote{Most MPI implementations provide additional backends for other networks.} \\
MPI+X & hybrid & explicit & yes & MPI \\
OmpSs & loop-, task-based & expl., \emph{impl.} & no & \\
OpenMP & loop-, task-based & expl., \emph{impl.} & no & \\
PaRSEC & task-based & implicit & yes & MPI \\
Realm & task-based & explicit & yes & GASNet \\
Regent & task-based & implicit & yes & GASNet \\
Spark & functional & implicit & yes & sockets \\
StarPU & task-based & implicit & yes & MPI \\
Swift/T & dataflow & implicit & yes & MPI \\
TensorFlow & dataflow & explicit & yes\footnote{Our evaluation only considers TensorFlow on a single node.} & sockets \\
X10 & place-based & explicit & yes & MPI\footnote{X10 also provides support for PAMI on supported networks.}
\end{tabular}

\caption{Systems for which we implemented Task Bench.\label{tab:systems}}
\vspace{-0.5cm}
\end{table}


We have implemented Task Bench in the 13 parallel and distributed
programming systems listed in Table~\ref{tab:systems}. We describe the
systems, and any salient details of their Task Bench implementations,
below.

One challenge in implementing Task Bench for such a wide variety of
systems is that the capabilities of the systems vary considerably. For
example, some systems are \emph{implicitly parallel}, and provide some
form of parallelism discovery from sequential programs, whereas others
are \emph{explicitly parallel} and require users to specify the
parallelism in the program. The Task
Bench implementations are intended to reflect how actual applications
would be written in the respective systems. Places where the Task
Bench implementations may be unidiomatic are noted below.

There are also aspects of some systems that we do not consider in our
evaluation. For example, a number of systems provide some level of
support for heterogeneous processors such as GPUs, but the support
varies so widely (if it exists at all) that it seems difficult to
compare systems with direct GPU support to those wihtout in an
apples-to-apples manner.

In all cases, members of the respective programming systems' teams
were consulted during the development and evaluation of the
corresponding Task Bench implementations. Where assistance was provided, the insights were invaluable
in helping to ensure that we provide the highest level of quality in the Task
Bench implementations for each system.

\subsection{Chapel}

Chapel~\cite{Chapel15} is a parallel programming language
with a \emph{multi-res\-o\-lu\-tion} approach, supporting parallelism at a variety of
levels. Chapel's core features are a partitioned global address space
(PGAS), data distributions, tasks,
synchronization primitives, and array promotion. For Task Bench, we target a low level of
Chapel, using explicit task instantiation (via
\lstinline[language=Chapel]{coforall}), bulk access to distributed
arrays for efficient data movement, and atomic integers for
synchronization.

\subsection{Charm++}

Charm++~\cite{Charmpp93} is an explicitly parallel actor-based programming system. Actors, or
\emph{chares}, are objects in their own address space.
Chares communicate data and synchronize via messages and can be moved
to balance load. Our Task Bench implementation uses a chare
array for the task graph, with one chare for each column. Messages implement dependencies; a task executes as soon as its
dependencies are all available. Collectives synchronize the
completion of the task graph execution.

% FIXME: Discuss SMP vs non-SMP versions

\subsection{MPI}

% FIXME: bulk synchronous or point-to-point

Our MPI~\cite{MPI} implementation of Task Bench represents the common
case of point-to-point communication with distinct computation and
communication phases. We experimented with a variety of implementation
strategies and found the best performing to be using
\lstinline[language=C++]{MPI_Isend} and
\lstinline[language=C++]{MPI_Irecv} to implement the communication
phase, posting receives before sends. Each task dependency maps to one
send/receive pair in MPI.

\subsection{OmpSs}

OmpSs~\cite{OmpSs11} is a programming model for loop- and task-based parallelism
that is source-compatible with OpenMP. Our Task Bench implementation
uses OpenMP 4.0-style task dependencies. Because OpenMP tasks have a fixed number of dependencies, we use a switch
statement to implement the dynamic dependencies required for Task
Bench. The implementation is otherwise straightforward.

% FIXME: More to say about OmpSs? Is there anything different vs the OpenMP implementation?

\subsection{OpenMP}

OpenMP~\cite{OpenMPSpec40} is the industry standard API for loop-based
parallelism on shared-memory systems, and supports task dependencies as of version 4.0. Our Task Bench implementation
for OpenMP is very similar to OmpSs and uses tasks with
task dependencies. We tested both GNU GOMP and Intel KMP
implementations of OpenMP and found KMP to be better performing.

% FIXME: keep this in sync with experiments

\subsection{PaRSEC}

PaRSEC is a task-based programming system supporting two distinct
programming models: \emph{parameterized task graphs}
(PTG)~\cite{PARSEC13} and \emph{dynamic task discovery}
(DTD)~\cite{PARSEC_DTD}.  PTG is a dataflow model in which programmers
write a concise, algebraic description of the tasks and dataflows in
the program. This compressed representation is expanded into a full
task graph by a source-to-source compiler.

In the DTD model, tasks are enumerated in program order (by
executing the program), and dependencies between tasks are
identified automatically based on the input and output data of tasks. 
A task depends on another task if it reads data written by the other task,
and the data is copied automatically if the two tasks are executed on
different nodes. To
improve horizontal scalability, the program is executed in
SPMD fashion on all nodes, and the user is responsible for eliding
tasks that are not directly connected to those that are to be executed
on the current node.

Task Bench uses a 2D block cyclic data distribution provided by PaRSEC. Tiles of data are 
distributed across nodes in pre-defined patterns,
and can be passed to tasks on other nodes, but
can only be written on the node where the data is originally
allocated. In our Task Bench implementation, each task writes to a
tile as output, and these tiles are passed as inputs to the
subsequent sets of tasks. As in the OmpSs and OpenMP implementations,
the task launch code uses a switch statement as the PaRSEC API assumes
tasks have a fixed number of dependencies.

\subsection{Realm}

Realm~\cite{Realm14} is an explicitly-parallel task-based programming
model used internally by Legion~\cite{Legion12} and
Regent~\cite{Regent15}. A Realm implementation of Task Bench is a limit study of what can be achieved with Legion or Regent.

Tasks in Realm are executed in a deferred manner, with dependencies
between tasks explicitly defined by \emph{events} passed from one task
to another. Realm's data model supports collections that live in a
specific memory. Data must be explicitly copied, and the copies
synchronized with tasks via events.

Realm also supports a \emph{subgraph} API that enables additional
optimizations. We report numbers for the subgraph version of Task
Bench as it provides better performance.

\subsection{Regent}

Regent~\cite{Regent15} is an implicitly-parallel task-based language
which implements the Legion programming model~\cite{Legion12}. We use
Regent rather than Legion directly because the Regent compiler
provides a critical optimization, \emph{control replication}, that
enables superior horizontal scalability~\cite{ControlReplication17}.

\subsection{Spark}

Spark~\cite{Spark10} is an implicitly-parallel programming model for
data analytics, widely used in industrial data center applications. As
one of the programming models in this study not originally intended
for scientific computing, Spark makes very different tradeoffs from
other systems under consideration, and therefore makes an interesting
point of comparison.

The core abstractions in Spark are functional operators such as map,
reduce, filter, join, etc. Functions in Spark operate on
\emph{resilient
  distributed datasets} (RDDs), which are globally-visible,
dynamically single-assignment data structures. One of Spark's primary
optimizations is the caching of RDDs to avoid unnecessary
disk traffic.

Though Spark has tasks internally, these are not exposed to the user,
so one of the challenges in developing a Task Bench implementation is
mapping the task graph to a set of operators that result in the
desired performance behavior. We use a combination of
\lstinline[language=Scala]{flatMap} and
\lstinline[language=Scala]{groupByKey} operations to generate task
dependencies, and then \lstinline[language=Scala]{mapPartitions} to execute a
series of tasks. We use an explicit hash partitioner to ensure that
Spark does not attempt to group multiple Task Bench tasks into a
single Spark task, as the tasks in Task Bench already represent coarse-grained units of work.

We performed extensive experiments to verify that no
extraneous factors interfered with our Spark measurements. We
completely disabled logging, ensured no RDDs are written
to disk, confirmed there is no measurable overhead due to JNI
calls from Java to C, and cross-checked results with known cases
that hit optimal task throughput in Spark, among other things.

\subsection{StarPU}

StarPU~\cite{StarPU11} is a task-based system that supports a \emph{sequential task flow} (STF)
programming model similar to PaRSEC's DTD. Our Task Bench implementation in
StarPU is very similar to PaRSEC, including the use of a switch
statement to handle dynamic numbers of task dependencies, and manual
elision of tasks on unrelated nodes to provide horizontal scalability.
We use a simplified version of the 2D block cyclic data distribution from the Chameleon 
project~\cite{Chameleon}.

\subsection{Swift/T}

Swift/T~\cite{Wozniak13} is a parallel scripting language intended
primarily for the composition of HPC workflows. Tasks in Swift/T can be written in any
programming language, and may even be themselves parallel. Swift/T
programs follow dataflow semantics, where every statement may
potentially execute in parallel as soon as its dependencies are
satisfied; statements only execute sequentially when explicitly
requested. The Swift/T compiler performs a number of optimizations to
improve performance of highly parallel programs~\cite{Armstrong14}.

Our Task Bench in Swift/T is straightforward, using Swift/T's dataflow
semantics to capture dependencies on other tasks.

\subsection{TensorFlow}

TensorFlow~\cite{TensorFlow15} is a programming system designed for
deep learning workloads. Although TensorFlow's API exposes
machine learning concepts, internally TensorFlow is a task graph execution engine, making it a good fit for
Task Bench. TensorFlow programs are written by creating nodes in the
task graph, and then connecting these nodes with edges to implement
the desired dataflow. Task graphs are composed in a high-level
language such as Python, but then serialized and run by an
execution engine written in C++.

Our Task Bench implementation in TensorFlow works by defining a custom
operator, which we instantiate for each point in the task graph. We
then define dataflow edges to correspond to the task dependencies.

\subsection{X10}

X10~\cite{X1005} is an explicitly parallel programming language for
place-based programming. The core features of X10 are \emph{places}
which represent distributed memories, a PGAS model where references to
remote objects can be held (but can only be dereferenced on the local
place), asynchronous tasks, and a place-changing construct to move the
execution of a task to a remote place. X10 also supports a variety of
synchronization primitives.

Our Task Bench implementation uses
\lstinline[language=X10]{Rail.asyncCopy} for efficient data movement
between places, along with place-changing and atomic integers for
synchronization. We use the native backend of X10, which compiles to
C++.

\section{Evaluation}
\label{sec:evaluation}

We present a comprehensive evaluation of our Task Bench implementations on up to 256
Haswell nodes of the Cori supercomputer~\cite{Cori}, a Cray XC40
machine. Cori Haswell nodes have 2 sockets with Intel Xeon E5-2698 v3
processors (a total of 32 physical cores per node), 128 GB RAM, and a
Cray Aries interconnect. We use GCC 7.3.0 for all Task Bench
implementations, and (where applicable) the system default MPI
implementation, Cray MPICH 7.7.3. Versions and flags for the
various systems are shown in Table~\ref{tab:flags}.

For GPU experiments we use the Piz Daint supercomputer~\cite{PizDaint}, a Cray XC50 with
one Intel Xeon E5-2690 v3 (12 physical cores) and one NVIDIA Tesla
P100 per node. We use GCC 6.2.0, Cray MPICH 7.7.2, and
CUDA 9.1.85.

\subsection{Compute Kernel Performance}
\label{subsec:peak-performance-and-efficiency}

In theory, any system should achieve peak performance if
the kernels are well-tuned and of sufficiently large granularity. In
practice, many subtle pitfalls of implementation or configuration can easily lead to poor performance. Verifying that peak performance is
achieved helps to ensure that evaluations of overhead and
efficiency are well-grounded.

Figure~\ref{fig:flops} shows the FLOP/s achieved with a compute-bound
kernel with varying problem sizes (simulated by running the kernel for varying numbers of iterations). This is the full version of
Figure~\ref{fig:flops-mpi}. Each data point in the graph is the mean of 5 runs, with Task Bench configured to execute 1000 time steps of the stencil pattern. In the best case, we measure peak FLOP/s of
$1.26 \times 10^{12}$, which compares favorably with the officially
reported number of $1.2 \times 10^{12}$ \cite{Cori}. For the purposes
of measuring efficiency, we use our empirically determined number as
the baseline for 100\% efficiency.

\begin{figure}[t]
\centering
\includegraphics[width=\columnwidth]{figs/task-bench-results/compute/flops_stencil.pdf}
\vspace{-0.7cm}
\caption{FLOPS vs problem size (stencil, 1 node). Higher is better.\label{fig:flops}}
\vspace{-0.45cm}
\end{figure}

\begin{figure}[t]
\centering
\includegraphics[width=\columnwidth]{figs/task-bench-results/compute/efficiency_stencil.pdf}
\caption{Task granularity vs efficiency (stencil, 1 node). Lower is better.\label{fig:efficiency}}
\end{figure}

\begin{figure}[t]
\centering
\includegraphics[width=\columnwidth]{figs/task-bench-results/memory/bytes_stencil.pdf}
\vspace{-0.7cm}
\caption{B/s vs problem size (stencil, 1 node). Higher is better.\label{fig:bytes}}
\vspace{-0.05cm}
\end{figure}


\begin{figure*}[t]

\subfloat[Stencil pattern.\label{fig:metg-compute-stencil}]{
\includegraphics[width=\columnwidth]{figs/task-bench-results/compute/metg_stencil.pdf}
}
\subfloat[Nearest pattern, 5 deps/task.\label{fig:metg-compute-nearest}]{
\includegraphics[width=\columnwidth]{figs/task-bench-results/compute/metg_nearest.pdf}
}

\subfloat[Spread pattern, 5 deps/task.\label{fig:metg-compute-spread}]{
\includegraphics[width=\columnwidth]{figs/task-bench-results/compute/metg_spread.pdf}
}
\subfloat[Nearest pattern, 5 deps/task, 4 independent graphs.\label{fig:metg-compute-4x-nearest}]{
\includegraphics[width=\columnwidth]{figs/task-bench-results/compute/metg_ngraphs_4_nearest.pdf}
}

\vspace{-0.15cm}
\caption{METG vs node count for different dependence patterns. Lower is better.\label{fig:metg-compute}}
\vspace{-0.25cm}
\end{figure*}


Most systems achieve or nearly achieve peak FLOP/s. Some
systems reserve a number of cores (usually 1 or 2) for internal
runtime usage; these systems take a minor hit in peak FLOP/s compared
to systems which share all cores between the application and runtime. Some of the
higher overhead systems struggle to achieve peak FLOP/s, though in most cases
the curves suggest that performance would continue to improve if we
were to run larger problem sizes. Unfortunately, the excessive
computational cost of running such tests makes this prohibitively
expensive. For example, the Spark job in
this case ran for over 6 hours.

Figure~\ref{fig:efficiency} plots efficiency (as a percentage of
peak FLOP/s) against task granularity. As described in Section~\ref{sec:metg}, this plot is
used to calculate METG(50\%). The
red, dashed line shows the point where 50\% efficiency is achieved.
In most cases, task granularity asymptotes prior to the 50\% efficiency point,
though some systems continue to improve as efficiency drops further. Accounting for this effect is one of the main arguments in
favor of METG with a reasonable efficiency threshold instead of
measuring task scheduling throughput of empty tasks
(effectively METG(0\%)). Measuring performance using empty tasks can
reward implementation strategies, such as devoting nearly 100\% of
system resources to the runtime system, that make no sense for real
applications.

\subsection{Memory Kernel Performance}

Figure~\ref{fig:bytes} shows performance with a memory-bound kernel. We measure a peak memory
bandwidth of 79~GB/s, using a working set size of 0.5 GB. As discussed in Section~\ref{sec:task-bench}, the kernels are designed to keep the working set constant as the number of iterations decrease to avoid noisy, superlinear effects in the results. For comparison, the
OpenMP-enabled STREAM benchmarks~\cite{STREAM} report up to 98~GB/s.

Not all cores are required to saturate memory bandwidth, reducing the
impact of reserving cores for internal use. Most systems hit 100\% of
peak, unlike the compute-bound case.

The remaining experiments use compute-bound kernels.

% FIXME: why compute-bound kernels?

\subsection{Baseline Overhead}

One question when considering different programming
systems is: How much overhead does the system add? This question is tricky to answer directly because some systems introduce
overhead \emph{inline} (i.e., by running system internal processes on
the same cores as application tasks), while other systems introduce
overhead \emph{out-of-line} (i.e., by dedicating one or more cores
solely to runtime use). Some systems, like Charm++, Realm, and Regent,
support both configurations.

To answer this question, we use the METG metric to determine the
smallest task granularity that can be executed at a given level of
efficiency as a proxy for overhead. Figure~\ref{fig:metg-compute}
shows how METG(50\%) varies with node count for a variety of
dependence patterns supported by Task Bench. METG(50\%) is calculated
separately at each node count, to distinguish runtime system behavior from
changes in communication latencies and topology when using
progressively larger portions of the machine.

We consider the following configurations of Task Bench:
Figure~\ref{fig:metg-compute-stencil} is a 1D stencil where each task
depends on 3 other tasks (including the same point in the previous
timestep). Figure~\ref{fig:metg-compute-nearest} is a pattern where
each task depends on 5 others, chosen to be as close as
possible. Figure~\ref{fig:metg-compute-spread} is a pattern where each
task depends on 5 others, spread as widely as possible. And
Figure~\ref{fig:metg-compute-4x-nearest} shows 4 identical copies of
the nearest dependence pattern executing concurrently.

We observe that the baseline overheads of different
systems vary by over 5 orders of magnitude. It is worth
remembering when considering this metric that this is a \emph{minimum}
effective task granularity. Therefore applications with an average
task granularity of \emph{at least} this value can usually be expected
to execute efficiently. Typical task granularities will
generally be determined by the application domain being
considered. Most notably, for large-scale data analytics workloads, the higher METG values observed for Spark and
TensorFlow are sufficient. In contrast, for high-performance
scientific simulations, task granularities in the millisecond range
are useful, as such applications communicate (e.g., for halo
exchanges) much more frequently.

The least complicated pattern (stencil) is most favorable
to MPI, as it provides no
opportunity for task parallelism. For the stencil pattern, the
dominating factor is the basic overhead of executing a task, which is
minimal for MPI as the Task Bench implementation simply executes tasks one after another in alternation with
communication phases. In contrast, the asynchronous execution
mechanisms of other systems are pure overhead in this scenario.
The gap between MPI and other systems shrinks as the complexity of the
communication pattern grows, and even reverses as task parallelism is
added in the form of multiple task graphs.

Spark, Swift/T and TensorFlow
are omitted from comparisons with more complicated dependencies, as the overheads of these systems require
excessive problem sizes (beyond what can be completed in 6
hours) to reach 50\% efficiency.

\subsection{Scalability}
\label{subsec:scalability}

METG is useful in part because it summarizes the overhead of each programming system in a single number. This makes it
possible to evaluate METG at different node counts (shown in
Figure~\ref{fig:metg-compute}) to see how it is impacted by changes in
communication topology and latency. A key finding is that
the systems with the smallest METG on one node have roughly an order
of magnitude higher METG at 256 nodes---increased communication latencies require significantly larger tasks to
achieve the same level of efficiency, so apparent differences in
runtime overhead at small node counts can matter much less or not at
all at larger node counts.

Most systems for HPC are highly scalable,
but this is not true of all the systems included in this
evaluation. Lower is better in Figure~\ref{fig:metg-compute}, and
flat is ideal. Lines that rise with node count indicate less than
ideal scaling. Most notably, Spark is primarily
intended industrial data center applications with task
granularities measured in tens of seconds. Spark uses a centralized
controller, which limits throughput, and this is visible in the figure
as the line for Spark immediately rises with node count. Keep in mind
that Spark is being evaluated here with a non-trivial dependence
pattern that is relatively unrepresentative of Spark's normal use
cases. Spark is more efficient with trivial parallelism, as described
in Section~\ref{subsec:number-of-dependencies}.

Implicitly parallel systems such as PaRSEC, StarPU and Regent that
rely on runtime analysis to build the DAG can suffer from
scalability bottlenecks if every node must consider the tasks
executing on all other nodes. PaRSEC DTD and StarPU mitigate this
partially by allowing the user to omit tasks not directly
dependent on those executed by a given node. However, this DAG
trimming approach requires dynamic checks that scale with
the number of nodes and thus limit scalability~\cite{PARSEC_DTD}. Compile-time analysis can
partially or fully mitigate this overhead. PaRSEC PTG improves over DTD 
by performing DAG expansion at compile time~\cite{PARSEC_DTD}, but
retains dynamic checks that limit scalability. PaRSEC shard includes additional manual optimizations over DTD, completely eliminating these dynamic checks. Regent
uses a compile-time optimization to improve
scalability~\cite{ControlReplication17}; the increase in METG beyond 16 nodes is due to a known bug in Realm barrier migration which the Realm Task Bench implementation is able to manually work around.

\subsection{Number of Dependencies}
\label{subsec:number-of-dependencies}

The number of dependencies per task has a strong influence on
overhead, as shown in
Figure~\ref{fig:radix}. This plot shows METG(50\%) for the nearest
dependence pattern, when varying the number of dependencies per task
from 0 to 9.

% FIXME: make sure these numbers are up to date for final paper

%% From 1 to 3
%% dependencies, the ratio varies from $1.0\times$ to $150\times$ (median
%% $2.3\times$).

The ratio in METG between 0 and 3 dependencies per task ranges from
$1.01\times$ to $250\times$ (median $2.9\times$). The difference is most pronounced in systems that
perform runtime work inline. For example, MPI achieves an METG of 390
ns with 0 dependencies, but this rises to 4.6 \textmu{}s with 3 dependencies,
a $12\times$ increase. This is unsurprising as in the case
of 0 dependencies, no \lstinline[language=C++]{MPI_Isend} calls are
issued at all, so MPI has nothing to do aside from executing kernels
as quickly as possible. Clearly, choosing a representative dependence
pattern is important when estimating the performance of a workload or
class of workloads.

\begin{figure}[t]
\centering
\includegraphics[width=\columnwidth]{figs/task-bench-results/radix/metg_nearest.pdf}
\vspace{-0.6cm}
\caption{METG vs deps/task (nearest, 1 node). Lower is better.\label{fig:radix}}
\vspace{-0.45cm}
\end{figure}


\zap{
\begin{figure}[t]
\centering
\includegraphics[width=\columnwidth]{figs/task-bench-results/compute/weak.pdf}
\vspace{-0.5cm}
% 262144 == 2^18
\caption{Weak scaling with problem size $2^{18}$ per node (stencil). Lower is better.\label{fig:weak-scaling}}
\vspace{-0.05cm}
\end{figure}

\begin{figure}[t]
\centering
\includegraphics[width=\columnwidth]{figs/task-bench-results/compute/strong.pdf}
\vspace{-0.5cm}
% 1048576 == 2^20
\caption{Strong scaling with problem size $2^{20}$ (stencil). Lower is better.\label{fig:strong-scaling}}
\vspace{-0.05cm}
\end{figure}

\subsection{Comparison with Weak and Strong Scaling}

Figures~\ref{fig:weak-scaling} and \ref{fig:strong-scaling} show
conventional strong and weak scaling of the stencil pattern for
comparison with the METG results in
Figure~\ref{fig:metg-compute-stencil}. At small node counts, where the
average task granularity is larger than the METG for most systems, the
performance is ideal. For several systems METG rises with node count,
and this becomes visible in the weak and strong scaling graphs as
non-ideal performance. Thus the shape of the weak and strong scaling
graphs effectively interpolates between ideal performance and the METG
curve depending on how close each system is to being limited by
overhead.
} % zap

\begin{figure*}[t!]

\subfloat[16 bytes per task dependency.\label{fig:efficiency-communication-16}]{
\includegraphics[width=\columnwidth]{figs/task-bench-results/communication/efficiency_nodes_64_comm_16.pdf}
}
\subfloat[256 bytes per task dependency.\label{fig:efficiency-communication-256}]{
\includegraphics[width=\columnwidth]{figs/task-bench-results/communication/efficiency_nodes_64_comm_256.pdf}
}

\subfloat[4096 bytes per task dependency.\label{fig:efficiency-communication-16}]{
\includegraphics[width=\columnwidth]{figs/task-bench-results/communication/efficiency_nodes_64_comm_4096.pdf}
}
\subfloat[65536 bytes per task dependency.\label{fig:efficiency-communication-256}]{
\includegraphics[width=\columnwidth]{figs/task-bench-results/communication/efficiency_nodes_64_comm_65536.pdf}
}

\caption{Task granularity vs efficiency for varying communication (spread pattern, 5 dependencies per task, 4 task graphs, 64 nodes). Lower is better.\label{fig:efficiency-communication}}
\end{figure*}


\begin{figure}[t]

\centering
\includegraphics[width=\columnwidth]{figs/task-bench-results/imbalance/efficiency_imbalance_2.0.pdf}

\vspace{-0.15cm}
\caption{Efficiency vs task granularity under load imbalance (nearest pattern, 5 dependencies per task, 4 task graphs, 1 node). Higher is better.\label{fig:efficiency-imbalance}}
\vspace{-0.1cm}
\end{figure}


\begin{figure}[t]
\centering
\includegraphics[width=\columnwidth]{figs/task-bench-results/cuda_compute/flops_normalized_stencil.pdf}
\vspace{-0.5cm}
\caption{GPU FLOP/s vs normalized problem size (stencil, 1 node). Higher is better.\label{fig:cuda-efficiency}}
\vspace{-0.05cm}
\end{figure}


\subsection{Communication Hiding}

Also of interest is the ability to
hide communication latency in the presence of task
parallelism. Figure~\ref{fig:efficiency-communication} plots efficiency with varying amounts of
communication, determined by the number
of bytes produced by each task (and therefore communicated with each
task dependency).

Asynchronous systems such as Charm++ demonstrate two benefits in
these plots. First, by overlapping communication with computation,
such systems execute smaller task granularities at higher
levels of efficiency compared to the MPI
implementations. Second, the asynchrony and scheduling flexibility from
executing multiple graphs also makes the curves smoother,
as spikes in latency due to interference from other jobs can be
mitigated, leading to more predictable performance, especially at
smaller message sizes.

\subsection{Load Imbalance}

One advantage of systems with asynchronous execution capabilities is
the ability to mitigate load imbalance with little or no additional programmer effort, especially in the presence of
task parallelism. To quantify this effect,
Figure~\ref{fig:efficiency-imbalance} plots task granularity vs
efficiency curves under load imbalance where each task's duration is multiplied by a uniform random variable between [0,~1). The task durations are generated with a deterministic
pseudo random number generator with a consistent seed to ensure
identical task durations for all systems.

The MPI implementation of Task Bench, with its distinct computation and communication phases,
suffers the most under load imbalance. The biggest
difference is at large task granularities, where the imbalance
effectively puts an upper bound on MPI efficiency. At smaller task
granularities the effect shrinks and may even reverse as systems hit
their fundamental limits due to overhead.

The remaining differences are due primarily to
different scheduling behaviors. The execution of 4
simultaneous task graphs only partially mitigates the
load imbalance between tasks. Systems that provide an
additional on-node work stealing capability (such as Chapel with the distrib scheduler) see additional gains in
efficiency at large task granularities. However, the use of
work-stealing queues can also impact throughput at small task
granularities. For example, Chapel's default (non-work-stealing) scheduler outperforms Chapel distrib at very small task granularities. We do not consider Charm++ load balancers because the imbalance is \emph{non-persistent} (i.e., timestep $t$ is uncorrelated with timestep $t+1$). We leave analysis of persistent load imbalance to future work.

\subsection{Heterogeneous Processors}

In order to determine the cost of scheduling tasks on the GPU, Figure~\ref{fig:cuda-efficiency} compares MPI
and MPI+CUDA on the Piz Daint supercomputer. The CUDA
compute kernel achieves a peak performance of $4.759 \times 10^{12}$
FLOP/s, which is very close to the officially reported number
$4.761 \times 10^{12}$. The CPU achieves $5.726 \times 10^{11}$
FLOP/s. Note that the kernels perform different numbers of
operations as the GPU requires more work to reach peak
performance. The x-axis in Figure~\ref{fig:cuda-efficiency} is
normalized to keep FLOPs constant for a given problem
size.

Our MPI+CUDA implementation uses an offload model in which data is copied to and from the GPU on each step. We test
two configurations: \lstinline{w1} uses 1 task per GPU, whereas
\lstinline{w4} overdecomposes by $4\times$, using 4 MPI
ranks per GPU to push work to the GPU in parallel.
\lstinline{w4} achieves higher FLOP/s but
drops more rapidly at smaller problem sizes, due to the overhead of
scheduling $4\times$ as many CUDA kernel launches. In either case, the
GPU requires more work to achieve high performance, and the overhead
of copying data dominates at small task
granularities, where the CPU achieves higher
performance. While Figure~\ref{fig:cuda-efficiency} is not couched in
terms of METG (as peak performance on CPU and GPU are very
different), the conclusion here is similar to
Section~\ref{subsec:scalability}: The cost of sending data and
tasks to the GPU imposes a higher task granularity to achieve
the same efficiency than running on the CPU, reducing the advantage at small task granularities of
very lightweight mechanisms such as those in MPI.

\section{Related Work}
\label{sec:related-work}

Parallel and distributed programming systems are most
commonly evaluated using proxy- or mini-apps, or
micro\-benchmarks. Mini-apps are explicitly derived from larger
applications and therefore have the advantage of bearing some
relationship to the original. This advantage typically does not hold
for microbenchmarks.

Though smaller than full applications, mini-apps can be challenging to
implement to a level of quality sufficient for conducting comparative
studies between programming systems. The largest study we know of,
for the mini-app LULESH~\cite{LULESH13}, compares 7 programming
systems (Chapel, Charm++, CUDA, Lizst, Loci, OpenMP, and MPI), each of
which require a wholly separate, tuned implementation (in contrast to
Task Bench). Other
studies usually lack a comprehensive evaluation, even if multiple
implementations are available:

\begin{itemize}

\item
The initial paper on PENNANT~\cite{PENNANT} includes an
implementation that supports MPI/OpenMP/MPI+OpenMP; follow-up papers present an implementation in
Regent~\cite{Regent15, ControlReplication17, LegionTracing18}.

\item
One follow-up paper for the mini-app CoMD describes a Chapel
implementation~\cite{CoMDChapel16} (comparison against reference
only). Additional follow-up papers consider aspects of the reference
implementation only~\cite{CoMDLoadImbalance17,
  CoMDThreadedModels14}.

\item
A report on the Mantevo project~\cite{Mantevo09} describes a number of
mini-apps, but only includes self-comparisons based on reference
implementations.

\item
A report on MiniAero~\cite{SandiaReportManyTaskRuntimes15} describes
four implementations of the mini-app, but only includes performance
results for three, and of those three only two can be compared in an
apples-to-apples manner as the last implementation uses a structured
rather than an unstructured mesh. Follow-up papers describe an
additional implementation in Regent~\cite{Regent15,
  ControlReplication17, LegionTracing18} (comparison against reference
only).

\end{itemize}

Microbenchmarks are often easier to implement, but may not be
representative of real-world applications. From the PRK
suite~\cite{PRK14}, the PRK Stencil code has been reasonably widely
implemented, and a follow-up paper~\cite{PRKRuntimes16} compares MPI
variants, SHMEM, UPC, Charm++, and Grappa. However, PRK Stencil is a
single, simple code involving 2 kernels (which combined fit in under 50 lines
of C++) and a halo exchange on a structured grid. The NAS benchmark suite~\cite{NAS91, NAS95} is also relatively widely studied with
implementations in OpenMP~\cite{NASOpenMP99}, MPI and
MPI+OpenMP~\cite{NASMPIOpenMP00}, and
Charm++~\cite{NASCharm96}. The benchmarks consist mostly of small
kernels for dense array computations. Neither the PRK nor the NAS
benchmark suite achieve the breadth of coverage, flexibility, or ease
of implementation of Task Bench. We believe the evidence for ease of
implementation is clear; Task Bench is implemented for more systems
than all previous mini-apps and microbenchmarks. Task Bench is also
not a fixed set of benchmarks, and it is even easier to use Task Bench
to generate a new benchmark to measure some aspect of system
performance.

System-specific benchmarks have been used to quantify specific aspects
of system performance, such as MPI communication or collective
latency~\cite{MPPTest99, MPIBench01}. These measurements typically do
not generalize beyond the immediate system they measure.

\textsc{coNCePTuaL}~\cite{Conceptual07} is a domain-specific language
for writing network performance tests. \textsc{coNCePTuaL} and Task
Bench both enable the easy creation of new benchmarks though
\textsc{coNCePTuaL} does so via scripting whereas Task Bench provides
a set of configurable parameters. \textsc{coNCePTuaL} also targets a
lower level of abstraction, optimized more for testing messaging
layers, whereas Task Bench is closer to application level and
therefore enables comparisons of a broader set of parallel and
distributed programming systems.

Limit studies of task scheduling throughput in various runtime systems
often make additional assumptions. A popular assumption is the use of
trivially parallel tasks~\cite{Canary16, Armstrong14}, which as shown
in Section~\ref{subsec:number-of-dependencies} underestimates (often
substantially) the cost of scheduling a task, and can also impact scalability.

\section{Conclusion}
\label{sec:conclusion}

Task Bench is a new approach for evaluating the performance of
parallel and distributed programming systems. By separating the
specification of a benchmark from implementations in various
programming systems, it is possible to explore a broad space of
application scenarios and to do so with a large number of programming
systems. Our experiments have enabled the following
insights:

\begin{itemize}

\item METG for current distributed programming systems varies by over
  5 orders of magnitude.  Clearly understanding the needed task
  granularity is an important consideration in choosing a programming
  system for a new application.

\item While some systems support task granularities as small as a few
  microseconds, this applies only to small CPU-based clusters. Adding
  accelerators or moving to clusters of hundreds of nodes raises the
  METG that any system can achieve by over an order of magnitude.

\item Systems that support asynchronous execution show benefits under
  load imbalanced workloads, or workloads with balanced computation
  and communication. However, these gains can be nullified by
  high baseline overheads.

\item Systems for large scale data analysis require very large tasks
  (in the tens of seconds) to scale beyond small numbers of nodes,
  reflecting the very coarse tasks and lack of need for strong scaling
  performance in current workloads.

\item The newer task-based systems have performance sufficient for
  weak scaling many HPC workloads, but more work is needed to strong
  scale the most demanding applications.

\end{itemize}

We have implemented Task Bench for a broad set of programming systems
spanning large scale data analytics and HPC. However, there are
systems not represented in our evaluation that would be interesting to
consider in the future. These include GASNet~\cite{GASNET07},
Habanero~\cite{Habanero11}, Hadoop~\cite{Hadoop},
HPX~\cite{Kaiser2014}, Nimbus~\cite{Nimbus17}, OCR~\cite{OCR14},
OpenSHMEM~\cite{OpenSHMEM10}, Ray~\cite{Ray18}, and UPC~\cite{UPC99}.

Not considered in our analysis is the impact of programming system
features on programmer productivity and performance portability. Most
applications do not operate at the absolute extreme of runtime-limited
performance, and thus may choose to trade overhead for better
usability. Our study helps to quantify the performance side of that
tradeoff so that users can be better informed and developers can see
the impact of features on the performance of their programming
systems.


\bibliographystyle{ACM-Reference-Format}
\bibliography{bibliography}

\end{document}
