\section{Task Bench}
\label{sec:task-bench}

\begin{table}[t]
\small

General options:

\vspace{0.25em}

\begin{tabular}{l | l | l}
Parameter & Values & Purpose \\
\hline

height & height of graph & number of timesteps \\
width & width of graph & degree of parallelism \\
dependence & trivial, stencil, etc. & dependence relation \\
\quad \raisebox{0.35ex}{$\llcorner$} radix & (for nearest pattern) & dependencies per task \\
kernel & compute, memory, etc. & type of kernel \\
output & bytes per dependency & degree of comm.
\end{tabular}

\vspace{1em}

Options specific to kernels:

\vspace{0.25em}

\begin{tabular}{l | l | l}
Parameter & Valid For & Purpose \\
\hline
iterations & compute, memory & task duration/problem size \\
span & memory & bytes used per task per iter. \\
scratch & memory & total working set size \\
imbalance & load imbalanced & degree of imbalance
\end{tabular}

\caption{Task Bench parameters.\label{tab:parameters}}
\vspace{-0.5cm}
\end{table}


To explore as broad a space of application scenarios as possible, Task
Bench provides a large number of configuration parameters. The most
important parameters are described in
Table~\ref{tab:parameters}. These parameters control the size and
structure of the task graph, the type and duration of the kernels
associated with each task, and the amount of data associated with
each dependence edge in the graph.

Task graphs are a combination of an iteration space (with a task for
each point in the space) with a dependence relation.
For simplicity, but without loss of generality, the iteration space in
Task Bench is constrained to be 2-dimensional, with time along
the vertical axis and parallel tasks along the
horizontal. Tasks may depend only on tasks from the immediately
preceding time step. Figure~\ref{fig:task-graphs} shows a number of sample task
graphs that can be implemented with Task Bench. Note that layout is
significant---in particular, column $i$ represents all tasks
with index $i$ in the iteration space over all the time
steps. Generally speaking each column will be
assigned to execute on a different processor core.

Dependencies between tasks are controlled by a configurable dependence
relation. The
dependence relation enumerates the set of tasks from the
previous time step each task depends on. This permits a wide variety
of dependence patterns to be implemented that are relevant to real
applications in high-performance scientific computing and data analysis: stencils,
sweeps, FFTs, trees, etc. Dependence patterns may also be
parameterized, such as picking the $K$ nearest neighbors, or $K$
distant neighbors. Table~\ref{tab:equations} shows equations for the
dependence relations of the patterns in Figure~\ref{fig:task-graphs},
where $t$ is timestep, $i$ is column, and $W$ is the width of the task
graph.

\begin{figure}[t]

\subfloat[Trivial.]{
\includegraphics[width=0.28\columnwidth]{figs/sample-task-graphs/trivial.pdf}
}
\hfill
\subfloat[Stencil.\label{fig:task-graphs-stencil}]{
\includegraphics[width=0.28\columnwidth]{figs/sample-task-graphs/stencil.pdf}
}
\hfill
\subfloat[FFT.]{
\includegraphics[width=0.28\columnwidth]{figs/sample-task-graphs/fft.pdf}
}

{\captionsetup[subfloat]{farskip=-0.15cm,captionskip=-0.15cm}
\subfloat[Sweep.]{
\includegraphics[width=0.28\columnwidth]{figs/sample-task-graphs/dom.pdf}
}
\hfill
\subfloat[Tree.]{
\includegraphics[width=0.32\columnwidth]{figs/sample-task-graphs/tree.pdf}
}
\hfill
\subfloat[Random.]{
\includegraphics[width=0.28\columnwidth]{figs/sample-task-graphs/random.pdf}
}
}

\vspace{-0.25cm}
\caption{Sample task graphs.\label{fig:task-graphs}}
\vspace{-0.2cm}
\end{figure}

\begin{table}[t]
\small
\begin{tabular}{l | l}
Pattern & Dependence Relation \\
\hline

% analytical style
%% Trivial & $D(i_t, j_{t-1}) := \operatorname{false}$ \\
%% Stencil & $D(i_t, j_{t-1}) := i_t - 1 \le j_{t-1} \le i_t + 1$ \\
%% FFT & $D(i_t, j_{t-1}) := j_{t-1} = i_t \vee j_{t-1} = i_t - 2^t \vee j_{t-1} = i_t + 2^t$ \\
%% Sweep & $D(i_t, j_{t-1}) := i_t - 1 \le j_{t-1} \le i_t$ \\
%% Tree & $D(i_t, j_{t-1}) := i_t - 1 \le j_{t-1} \le i_t + 1$ \\

% set style
Trivial & $D(t, i) := \emptyset$ \\
Stencil & $D(t, i) := \{ i, i - 1, i + 1 \}$ \\
FFT & $D(t, i) := \{ i, i - 2^t, i + 2^t \}$ \\
Sweep & $D(t, i) := \{ i, i - 1 \}$ \\
Tree & $D(t, i) := \left\{ \!\!\!\! \arraycolsep=4pt\begin{array}{l l} \{ i - 2^{-t}W(i \operatorname{mod} 2^{-t + 1}W) \} & \text{if } t \le \operatorname{log}_2 W \\ \{ i, i + 2^{t-1}W^{-1} \} & \text{otherwise} \end{array} \right. \!\!\!\! $ \\
Rand. & $D(t, i) := \{ i | 0 \le i < W \wedge \operatorname{random}() < 0.5  \}$ \\
\end{tabular}

\caption{Dependence relations for sample task graphs.\label{tab:equations}}
\vspace{-0.5cm}
\end{table}


Despite its generality, Task Bench is easy to implement, making it
tractable to develop a suite of high-quality implementations. The central aspects of Task Bench, such as generating
task graphs and enumerating dependencies, are encapsulated in a core
library that is shared among all the Task Bench implementations. The
core library also includes implementations of the kernels, ensuring
that the kernels are identical in all systems, eliminating a potential
source of performance disparity that can be a pitfall for
implementations of mini-apps. Finally, the core library manages
parsing input parameters and displaying results,
ensuring that all implementations behave uniformly and can
be scripted consistently. Because much of the functionality needed for
a Task Bench implementation is in the core library, implementations of
Task Bench are small: Our 15 Task Bench implementations range from 88
to 1500 lines, with several hundred lines being
typical. Listings~\ref{lst:compute-kernel} and \ref{lst:code-sample}
show excerpts from the Task Bench core, and the Dask implementation,
respectively. Only the code in Listing~\ref{lst:code-sample} is
implemented for each system, minimizing the work required for each additional system.

\begin{lstlisting}[language=C,caption={Core API implementation of compute kernel.},label={lst:compute-kernel},style=codeblock,float]
void compute_kernel(long iterations) {
  double A[64];
  for (int i = 0; i < 64; i++) A[i] = 1.2345;
  for (long iter = 0; iter < iterations; iter++)
    for (int i = 0; i < 64; i++)
      A[i] = A[i] * A[i] + A[i];
}
\end{lstlisting}

%% \begin{lstlisting}[language=Python,caption={Excerpt from Task Bench implementation in Dask.\label{lst:code-sample}},float]
def execute_point(g, t, p, scratch, inputs):
   output = make_buffer(g.output_bytes_per_task)
   # call Task Bench Core API...
   return output, scratch

def execute_task_graph(g):
   scratch = [
      make_buffer(g.scratch_bytes_per_task)
      for _ in range(0, g.width)
   ]
   outputs = []
   last_row = None
   for t in range(0, g.timesteps):
      row = []
      for p in range(0, g.width):
         inputs = []
         for dep in dependencies(g, t, p):
            inputs.append(last_row[dep])
         output, scratch[p] = execute_point(
            g, t, p, scratch[p], inputs)
         row.append(output)
         outputs.append(output)
      last_row = row
   return outputs
\end{lstlisting}

\begin{lstlisting}[language=C++,caption={Excerpt from Task Bench implementation in MPI.},label={lst:code-sample},style=codeblock,float]
void execute_task_graph(Graph g) {
  char *output = (char *)malloc(g.output_bytes);
  char *scratch = (char *)malloc(g.scratch_bytes);
  char **inputs = (char **)malloc(/*...*/);
  long rank;
  // initialize data structures...

  std::vector<MPI_Request> requests;
  for (long row = 0; row < g.height; ++row) {
    if (g.contains_point(row, rank)) {
      long idx = 0;
      requests.clear();

      for (long dep : g.deps(row, rank)) {
        MPI_Request req;
        MPI_Irecv(
          inputs[idx], g.output_bytes, MPI_BYTE,
          dep, 0, MPI_COMM_WORLD, &req);
        requests.push_back(req);
        idx++;
      }

      for (long dep : g.reverse_deps(row, rank)) {
        MPI_Request req;
        MPI_Isend(
          output, g.output_bytes, MPI_BYTE,
          dep, 0, MPI_COMM_WORLD, &req);
        requests.push_back(req);
      }

      MPI_Waitall(
        requests.size(), requests.data(),
        MPI_STATUSES_IGNORE);

      g.execute_point(
        row, rank,
        output, g.output_bytes_per_task,
        inputs, input_bytes, idx,
        scratch, g.scratch_bytes_per_task);
    }
  }
}
\end{lstlisting}


The Task Bench core library is fully
validating. Because the task graph configuration is explicitly
represented (though unmaterialized) in Task Bench, this representation
can be queried to determine exactly what dependencies a task should
expect. The output of every task in Task Bench is unique,
and all inputs are verified. An assertion is thrown if validation
fails. These checks ensure that every execution of Task Bench, if it
completes successfully, is correct. An evaluation of the performance impact of
validation showed it to be less than 3\% at the smallest task
granularities in any Task Bench implementation, with a negligible
effect on overall results.

Task Bench provides two main kernels that can be called from tasks:
compute- and memory-bound. The compute-bound
kernel executes a tight loop and is hand-written using AVX2 FMA
intrinsics. The memory-bound kernel performs sequential reads and
writes over an array, again with AVX2
intrinsics. The duration of both kernels can be configured by setting
the number of iterations to execute; we use this ability to simulate
the effects of varying application problem sizes. The memory-bound
kernel is carefully written to keep the working set size constant as
the number of iterations decreases, to avoid unwanted speedups due to
cache effects.
