\section{METG}
\label{sec:metg}

\begin{figure}[t]
\centering
\includegraphics[width=\columnwidth]{figs/task-bench-results/compute/flops_stencil.pdf}
\caption{FLOPS vs problem size (stencil pattern).\label{fig:flops}}
\end{figure}

\begin{figure}[t]
\centering
\includegraphics[width=\columnwidth]{figs/task-bench-results/compute/efficiency_stencil.pdf}
\caption{Task granularity vs efficiency (stencil pattern).\label{fig:efficiency}}
\end{figure}


Minimum effective task granularity (METG) is defined intuitively as
the smallest average task granularity such that the application
achieves overall efficiency exceeding some threshold. For example, in
a compute-bound application efficiency might be measure as the
percentage of the available FLOPS achieved in a given application
run. On a machine with a peak capability of 1.2 TFLOPS per node,
METG(50\%) would correspond to smallest task granularity achieved
while maintaing at least 0.6 TFLOPS. Absolute efficiency (i.e., FLOPS
or bytes per second) is preferrable, but parallel efficiency (relative
to a sufficiently large problem size) can also be used in applications
which are not amenable to being characterized by peak resource
utilization.

Figure~\ref{fig:flops} shows the way in which METG is calculated. The
application is run with a large enough problem size to ensure that it
achieves peak utilization of the machine. This confirms that the
application is properly configured and that the choice of efficiency
metric is achievable. Once this is established, the problem size is
shrunk (in a manner analogous to what happens on a node during a
strong scaling run). The expectation is as problem size shrinks, the
efficiency will begin to drop and eventually approach zero. As seen in
Figure~\ref{fig:flops}, the systems with lower overheads will maintain
higher efficiency at smaller problem sizes compared to systems with
higher overheads.

To calculate METG, this data is replotted along the axes of efficiency
(i.e., as a percentage of the peak FLOPS achieved) and task
granularity (i.e., $\text{wall time} \times \text{number of
  cores}/\text{number of tasks}$). Figure~\ref{fig:efficiency} shows
the replotted data from above. In this chart we can see that
efficiency starts near 100\% for the vast majority of systems, and
that initially task granularity drops quickly with minimal change in
efficiency. However as task granularity shrinks, efficiency begins to
drop as well, leading most systems to reach a plateau. Intuitively,
this plateau represents the fundamental baseline overhead of the
system, the point beyond which the system simply cannot execute tasks
efficiently.

METG is thus the result of drawing a vertical line through the chart
in Figure~\ref{fig:efficiency} at the desired level of efficiency. For
example, at 50\% efficiency, MPI achieves an average task granularity
of 4.5 us while Charm++ achieves an average task granularity of 12
us. Thus the METG(50\%) of MPI and Charm++ are 4.5 and 12 us,
respectively, for this application. The number 50\% is chosen
arbitrarily, but is a reasonable choice for this application because
it represents a point at which most systems have plateaued, yet the
overall efficiency is still reasonable. Lower values would likely not
be reasonable, because in most cases operators of computing facilities
prefer to maintain higher levels of utilization of the machines. And
higher values would misrepresent the performance of more systems.

METG is an appealing metric because it has a well-understood
relationship with quantities of interest to application developers,
namely weak and strong scaling. METG corresponds to the smallest
problem size that can be weak scaled to a given node count with a
given level of efficiency. Larger problem sizes would correspond with
larger tasks, and therefore have lower overhead relative to useful
computation. Similarly, METG corresponds to the point at which strong
scaling can be expected to stop; as strong scaling runs start with a
larger problem size and progressively shrink it until overheads begin
to dwarf the gains due to continued scaling.

The METG metric has one particularly useful property in comparison to
weak and strong scaling. Because METG is measured ``in place'' (i.e.,
without changing the number of nodes or cores available to the
application), it is possible to use METG to isolate effects that are
due to shrinking problem size from effects that are due to e.g.,
increased communication latencies as progressively larger portions of
the machine are being used. Section~\ref{sec:evaluation} contains a
comprehensive evaluation of METG including how it changes with node
count.

% talk about TPS / MTG ?
