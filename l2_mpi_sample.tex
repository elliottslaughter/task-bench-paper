\begin{lstlisting}[language=C++,caption={Excerpt from Task Bench implementation in MPI.},label={lst:code-sample},style=codeblock,float]
void execute_task_graph(Graph g) {
  char *output = (char *)malloc(g.output_bytes);
  char *scratch = (char *)malloc(g.scratch_bytes);
  char **inputs = (char **)malloc(/*...*/);
  long rank;
  // initialize data structures...

  std::vector<MPI_Request> requests;
  for (long t = 0; t < g.height; ++t) {
    if (g.contains_point(t, rank)) {
      long idx = 0;
      requests.clear();

      for (long dep : g.deps(t, rank)) {
        MPI_Request req;
        MPI_Irecv(inputs[idx], g.output_bytes, MPI_BYTE,
          dep, 0, MPI_COMM_WORLD, &req);
        requests.push_back(req);
        idx++;
      }

      for (long dep : g.reverse_deps(t, rank)) {
        MPI_Request req;
        MPI_Isend(output, g.output_bytes, MPI_BYTE,
          dep, 0, MPI_COMM_WORLD, &req);
        requests.push_back(req);
      }

      MPI_Waitall(requests.size(), requests.data(),
        MPI_STATUSES_IGNORE);

      g.execute_point(t, rank, output, g.output_bytes_per_task,
        inputs, input_bytes, idx,
        scratch, g.scratch_bytes_per_task);
    }
  }
}
\end{lstlisting}
