\section{Conclusion}
\label{sec:conclusion}

Task Bench is a new approach for evaluating the performance of
parallel and distributed programming systems. By separating the
specification of a benchmark from implementations in various
programming systems, it is possible to explore a broad space of
application scenarios and to do so with a large number of programming
systems. Our experiments have enabled the following
insights:

\begin{itemize}

\item METG for current distributed programming systems varies by over
  5 orders of magnitude.  Clearly understanding the needed task
  granularity is an important consideration in choosing a programming
  system for a new application.

\item While some systems support task granularities as small as a few
  microseconds, this applies only to small CPU-based clusters. Adding
  accelerators or moving to clusters of hundreds of nodes raises the
  METG that any system can achieve by over an order of magnitude.

\item Systems that support asynchronous execution show benefits under
  load imbalanced workloads, or workloads with balanced computation
  and communication. However, these gains can be easily nullified by
  high baseline overheads.

\item Systems for large scale data analysis require very large tasks
  (in the tens of seconds) to scale beyond small numbers of nodes,
  reflecting the very coarse tasks and lack of need for strong scaling
  performance in current workloads.

\item The newer task-based systems have performance sufficient for
  weak scaling many HPC workloads, but more work is needed to strong
  scale the most demanding applications.

\end{itemize}

We have implemented Task Bench for a broad set of programming systems
spanning large scale data analytics and HPC. However, there are
systems not represented in our evaluation that would be interesting to
consider in the future. These include GASNet~\cite{GASNET07},
Habanero~\cite{Habanero11}, Hadoop~\cite{Hadoop},
HPX~\cite{Kaiser2014}, Nimbus~\cite{Nimbus17}, OCR~\cite{OCR14},
OpenSHMEM~\cite{OpenSHMEM10}, Ray~\cite{Ray18}, and UPC~\cite{UPC99}.

Not considered in our analysis is the impact of programming system
features on programmer productivity and performance portability. Most
applications do not operate at the absolute extreme of runtime-limited
performance, and thus may choose to trade overhead for better
usability. Our study helps to quantify the performance side of that
tradeoff so that users can be better informed and developers can see
the impact of features on the performance of their programming
systems.
