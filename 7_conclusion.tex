\section{Conclusion}
\label{sec:conclusion}

Task Bench is a promising new approach for evaluating the performance
of parallel and distributed programming systems. We have seen that
systems with support for asynchronous (and/or task-based) programming
appear to show benefits with more complicated dependency patterns,
higher node counts (because of higher communication latencies) and
more communication (because of ability to overlap it with
computation). However, all these benefits can be eclipsed if baseline
runtime overheads are too high. The current generation of task-based
systems (PaRSEC, Realm, Regent, and StarPU) show higher overheads
than traditional HPC systems, indicating that there continues to be
work to be done in this area.

Not considered in our analysis is the impact of programming system
features on programmer productivity and performance portability. Most
applications do not operate at the absolute extreme of runtime-limited
performance, and thus may choose to trade overhead for better
usability. Our study helps to quantify the performance side of that
tradeoff so that users can be better informed and developers can see
the impact of features on the performance of their programming
systems.
