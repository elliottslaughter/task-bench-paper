\section{Conclusion}
\label{sec:conclusion}

Task Bench is a new approach for evaluating the performance
of parallel and distributed programming systems. We have seen that
systems vary by orders of magnitude in runtime overhead, as measured
by METG. Widely used systems in industry have METG measured in
seconds, reflecting the very coarse tasks and lack of need for strong
scaling performance in current workloads. At the other extreme,
traditional HPC systems have METG in the range of 10-100 \textmu{}s,
necessary for strong scaling to small task granularities, though we
have shown that METG also increases significantly for these systems
with the size of the cluster and the use offload accelerators. The newer
task-based systems, with METG in the range of 100 \textmu{}s to 1 ms
have performance sufficient for weak scaling many HPC workloads, but
more work is needed to strong scale the most demanding applications.

Not considered in our analysis is the impact of programming system
features on programmer productivity and performance portability. Most
applications do not operate at the absolute extreme of runtime-limited
performance, and thus may choose to trade overhead for better
usability. Our study helps to quantify the performance side of that
tradeoff so that users can be better informed and developers can see
the impact of features on the performance of their programming
systems.
