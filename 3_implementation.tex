\section{Implementations}
\label{sec:implementation}

\begin{table}[t]
\begin{tabular}{l | l | l | l | l}
System & Paradigm & Parallelism & Distrib. & Network \\
\hline
Chapel & multi-resolution & \emph{expl.}, impl. & yes & uGNI\footnote{Chapel uses GASNet to support non-Cray networks.} \\
Charm++ & actor model & explicit & yes & uGNI\footnote{Charm++ provides additional backends for other networks.} \\
Dask & task-based & implicit & yes & sockets \\
MPI & message passing & explicit & yes & uGNI\footnote{Most MPI implementations provide additional backends for other networks.} \\
MPI+X & hybrid & explicit & yes & MPI \\
OmpSs & loop-, task-based & expl., \emph{impl.} & no & \\
OpenMP & loop-, task-based & expl., \emph{impl.} & no & \\
PaRSEC & task-based & implicit & yes & MPI \\
Realm & task-based & explicit & yes & GASNet \\
Regent & task-based & implicit & yes & GASNet \\
Spark & functional & implicit & yes & sockets \\
StarPU & task-based & implicit & yes & MPI \\
Swift/T & dataflow & implicit & yes & MPI \\
TensorFlow & dataflow & explicit & yes\footnote{Our evaluation only considers TensorFlow on a single node.} & sockets \\
X10 & place-based & explicit & yes & MPI\footnote{X10 also provides support for PAMI on supported networks.}
\end{tabular}

\caption{Systems for which we implemented Task Bench.\label{tab:systems}}
\vspace{-0.5cm}
\end{table}


We have implemented Task Bench in the 15 parallel and distributed
programming systems listed in Table~\ref{tab:systems}. We describe the systems, and any
salient details of their Task Bench implementations, below.

One challenge in targeting such a wide variety of
systems is that the capabilities of the systems vary considerably. For
example, some systems are \emph{implicitly parallel}, and provide some
form of parallelism discovery from sequential programs, whereas others
are \emph{explicitly parallel} and require users to specify the
parallelism in the program. For systems that provide both implicit and explicit parallelism, the form of parallelism used in Task Bench is emphasized in Table~\ref{tab:systems}.

%% The Task
%% Bench implementations are intended to reflect how actual applications
%% would be written in the respective systems. Places where the Task
%% Bench implementations may be unidiomatic are noted below.

%% There are also aspects of some systems that we do not consider. For
%% example, tasks may run on CPUs, GPUs, or any execution resource
%% supported by a programming system. However, the level of support for
%% heterogeneous processors varies widely across programming systems (if it exists at all), so
%% in order to include as many systems as possible in our comparisons, we
%% evaluate Task Bench only on CPUs in this paper.

In all cases, members of the programming systems' teams
were consulted in the development and evaluation of the
corresponding Task Bench implementations. Where assistance was provided, the insights helped ensure that we provide the highest quality implementations for each system.


\subsection{Chapel}

Chapel~\cite{Chapel15} is a parallel programming language
with a \emph{multi-res\-o\-lu\-tion} approach, supporting parallelism at a variety of
levels. Chapel's core features are a partitioned global address space
(PGAS), data distributions, tasks,
synchronization primitives, and array promotion. For Task Bench, we target a low level of
Chapel, using explicit task instantiation (via
\lstinline[language=Chapel]{coforall}), bulk access to distributed
arrays for efficient data movement, and atomic integers for
synchronization.

\subsection{Charm++}

Charm++~\cite{Charmpp93} is an explicitly parallel actor-based programming system. Actors, or
\emph{chares}, are objects in their own address space.
Chares communicate data and synchronize via messages and can be moved
to balance load. Our Task Bench implementation uses a chare
array for the task graph, with one chare for each column. Messages implement dependencies; a task executes as soon as its
dependencies are all available.

%% Collectives synchronize the
%% completion of the task graph execution.

% FIXME: Discuss SMP vs non-SMP versions

\subsection{Dask}

Dask~\cite{Dask15} is an implicitly parallel task-based programming system for large
scale data analytics with Python. Dask provides abstractions over
distributed NumPy arrays and Pandas dataframes, and also a lower-level
interface for launching tasks. We use the lower-level interface to
implement Task Bench.

\subsection{MPI}

Our MPI~\cite{MPI} implementation of Task Bench represents the common
case of point-to-point communication with distinct computation and
communication phases. We experimented with a variety of implementation
strategies and found the best performing to be using
\lstinline[language=C++]{MPI_Isend} and
\lstinline[language=C++]{MPI_Irecv} to implement the communication
phase, posting receives before sends. Each task dependency maps to one
send/receive pair in MPI. We also provide a bulk synchronous
implementation which enforces the boundary between communication and
computation with \lstinline[language=C++]{MPI_Barrier}.

\subsection{MPI+X}

We provide two MPI+X implementations of Task Bench to evaluate the
performance of hierarchical programming models. Our MPI+OpenMP
implementation uses forall-style parallel loops to execute tasks,
interleaved with MPI point-to-point communication as above. Our
MPI+CUDA implementation follows an offload model where data is copied
to and from the GPU on every timestep.

\subsection{OmpSs}

OmpSs~\cite{OmpSs11} is a programming model for loop- and task-based parallelism
that is source-compatible with OpenMP. Our Task Bench implementation
uses OpenMP 4.0-style task dependencies. Because OpenMP tasks have a fixed number of dependencies, we use a switch
statement to implement the dynamic dependencies required for Task
Bench. The implementation is otherwise straightforward.

% FIXME: More to say about OmpSs? Is there anything different vs the OpenMP implementation?

\subsection{OpenMP}

OpenMP~\cite{OpenMPSpec40} is the industry standard API for loop-based
parallelism on shared-memory systems, and supports task dependencies as of version 4.0. Our Task Bench implementation
for OpenMP is similar to OmpSs and uses
task dependencies. We tested with GNU GOMP and Intel KMP
and found KMP to be better performing.

\subsection{PaRSEC}

PaRSEC is a task-based programming system supporting two distinct
programming models: \emph{parameterized task graphs}
(PTG)~\cite{PARSEC13} and \emph{dynamic task discovery}
(DTD)~\cite{PARSEC_DTD}.  PTG is a dataflow model in which programmers
write a concise, algebraic description of the tasks and dataflows in
the program. This compressed representation is expanded into a full
task graph by a source-to-source compiler.

In the DTD model, tasks are enumerated in program order (by
executing the program), and dependencies between tasks are
identified automatically based on the input and output data of tasks. 
A task depends on another task if it reads data written by the other task,
and the data is copied automatically if the two tasks are executed on
different nodes. To
improve scalability, the program is executed in
SPMD fashion on all nodes, and the user is responsible for eliding
tasks that are not directly connected to those that are to be executed
on the current node.

In our Task Bench implementation, we evaluate both PTG and DTD models,
as well as a third option based on DTD that has additional manual
optimizations applied to further reduce dynamic checks for what tasks
should be executed on a node, which further improves
scalability.

%% Task Bench uses a 2D block cyclic data distribution provided by PaRSEC. Tiles of data are 
%% distributed across nodes in pre-defined patterns,
%% and can be passed to tasks on other nodes, but
%% can only be written on the node where the data is originally
%% allocated. In our Task Bench implementation, each task writes to a
%% tile as output, and these tiles are passed as inputs to the
%% subsequent sets of tasks. As in the OmpSs and OpenMP implementations,
%% the task launch code uses a switch statement as the PaRSEC API assumes
%% tasks have a fixed number of dependencies.

\subsection{Realm}

Realm~\cite{Realm14} is an explicitly-parallel task-based programming
model used internally by Legion~\cite{Legion12} and
Regent~\cite{Regent15}. A Realm implementation of Task Bench is a limit study of what can be achieved with Legion or Regent.

Tasks in Realm are asynchronous, with dependencies
defined by \emph{events} passed from one task
to another. Realm's data model supports collections that live in a
specific memory. Data must be explicitly copied, and the copies
synchronized with tasks via events.
Realm also supports a \emph{subgraph} API that enables additional
optimizations. We use subgraphs in the Task
Bench implementation.

\subsection{Regent}

Regent~\cite{Regent15} is an implicitly-parallel task-based language
which implements the Legion programming model~\cite{Legion12}. We use
Regent rather than Legion directly because the Regent compiler
provides a critical optimization that
improves scalability~\cite{ControlReplication17}.

\subsection{Spark}

Spark~\cite{Spark10} is an implicitly-parallel programming model for
large scale data analytics.
The core abstractions in Spark are functional operators such as map,
reduce, and join. Data in Spark are stored in
\emph{resilient
  distributed datasets} (RDDs): globally-visible,
dynamically single-assignment data structures. Spark caches RDDs in memory to avoid unnecessary
disk traffic.

Spark has tasks internally but they are not exposed to the user,
so a Task Bench implementation must map the task graph to a set of operators that result in the
desired execution. We use
\lstinline[language=Scala]{flatMap} and
\lstinline[language=Scala]{groupByKey} to generate task
dependencies, and then \lstinline[language=Scala]{mapPartitions} to execute a
series of tasks. An explicit hash partitioner ensures that
Spark does not attempt to group multiple Task Bench tasks into a
single Spark task, as tasks in Task Bench already represent coarse-grained units of work.

We performed extensive experiments to verify that no
extraneous factors interfered with our Spark measurements. We
disabled logging, ensured no RDDs are written
to disk, confirmed there is no measurable overhead due to JNI
calls from Java to C or due to the serialization of data structures, and cross-checked results with known cases
that hit optimal task throughput in Spark.

\subsection{StarPU}

StarPU~\cite{StarPU11} is a task-based system that supports a \emph{sequential task flow} (STF)
programming model similar to PaRSEC's DTD. Our Task Bench implementation in
StarPU is very similar to PaRSEC.
%% We use a simplified version of the 2D block cyclic data distribution from the Chameleon 
%% project~\cite{Chameleon}.

\subsection{Swift/T}

Swift/T~\cite{Wozniak13} is a parallel scripting language intended
primarily for the composition of HPC workflows. Swift/T
programs follow dataflow semantics, where every statement may
potentially execute in parallel as soon as its dependencies are
satisfied; statements only execute sequentially when explicitly
requested. The Swift/T compiler performs a number of optimizations to
improve performance of highly parallel programs~\cite{Armstrong14}. Our Task Bench in Swift/T is straightforward, using Swift/T's dataflow
semantics to capture dependencies on other tasks.

\subsection{TensorFlow}

TensorFlow~\cite{TensorFlow15} is a programming system designed for
deep learning workloads. Although TensorFlow's API exposes
machine learning concepts, internally TensorFlow is a task graph execution engine, making it a good fit for
Task Bench. TensorFlow programs operate via explicit graph construction. Task graphs are composed in Python and run by an
execution engine written in C++. Our Task Bench implementation is straightforward.

\subsection{X10}

X10~\cite{X1005} is an explicitly parallel programming language for
place-based programming. The core features of X10 are \emph{places}
which represent distributed memories, a PGAS model where references to
remote objects can be held (but can only be dereferenced on the local
place), asynchronous tasks, and a place-changing construct to move the
execution of a task to a remote place. X10 also supports a variety of
synchronization primitives.

Our Task Bench implementation uses
\lstinline[language=X10]{Rail.asyncCopy} for efficient data movement
between places, along with place-changing and atomic integers for
synchronization. We use the native backend of X10, which compiles to
C++.
