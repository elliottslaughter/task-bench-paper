\section{Introduction}
\label{sec:introduction}

Challenges in heterogeneous and extreme-scale parallel and distributed
computing have lead to the rise of a wide variety of new programming
models, languages and runtime systems. These systems often provide
capabilities such as asynchronous execution of units of work, called
\emph{tasks}, coordinated in some manner with data movement and
synchronization. But while the features of these systems are widely
written about in the literature, a large and often difficult to answer
question is how different systems will perform for a given
application.

One approach to answering this question has focused around the
development of \emph{proxy-} or \emph{mini-apps}. So called because
they embody the logic of a larger application, but in reduced form,
mini-apps are popular because they can provide insight without
incurring the costly expense of developing a full-blown application
code. However, despite the name, mini-apps still requite significant
investment to develop, especially to a level of quality needed to reach
a publishable state. In many cases, the effort to tune for performance
exceeds the effort to develop the mini-app in the first place. This
can be exacerbated if the system in question is undergoing concurrent
development. As a result, implementations of mini-apps, though
well-intentioned, often do not reach a level of maturity required to
perform a broad comparison between systems, and as a result there are
few comparative studies involving multiple programming systems
that reach publication.

A compounding issue is one of methodology: how does one predict how
the performance observed for a mini-app will generalize to a full
application, even one closely related? Part of the problem here is
that weak and strong scaling, which are by far the most commonly
reported metrics for performance of both mini- and full applications,
do nothing to elucidate the performance envelop of a particular
system. Weak scaling is notorious for being able to hide arbitrary
amounts of overhead, as long as the problem size is sufficiently
large. And strong scaling runs the risk of entangling unrelated
effects: namely, the baseline overhead of a given system, with the
growth of communication latencies, and potentially the saturation of
global resources such as communication and I/O bandwidth, when using
progressively larger portions of the machine.

To address the first of these issues, we present Task Bench, a
\emph{parameterized} benchmark designed for exploring the performance
behaviors of programming systems under a variety of conditions. A
critical property of Task Bench is that, while powerful, it is simple
enough to be implemented in a wide variety of systems with reasonable
effort, to a level of quality necessary for comparative study.

Task Bench consists of a configurable \emph{task graph} with tasks for
each point in an iteration space. Tasks are connected by dependencies
as determined by a \emph{dependence relation}. This design permits a
very concise description of a wide variety of patterns relevant to
scientific computing: trivial parallelism, halo exchanges (such as
seen in structured and unstructured mesh codes), sweeps (such as used
in the discrete ordinates method of radiation simulation), FFTs, trees
(for divide and conquer algorithms), and so on. The tasks at each
point in the task graph can be configured to execute kernels with a
variety of computational properties, including compute- and
memory-bound operations of varying durations. The dependencies between
tasks can be configured to carry payloads of varying size, permitting
the design of communication-bound cases. Finally, multiple
(potentially heterogeneous) task graphs can be executed concurrently
to introduce task parallelism into the workload. Together, these
design elements permit the exploration of a broad space of application
performance behaviors.

To resolve the methodological issue, we introduce a new metric called
\emph{minimum effective task granularity} (METG). Intuitively, METG is
the smallest task granularity which can be executed while maintaining
a given level of computational efficiency (e.g., as measured by the
percentage of peak FLOPS achieved). METG captures the essence of what
is important in a weak or strong scaling study: that is, the behavior
at the limit of scalability. For weak scaling, METG corresponds
directly to the smallest problem size that can be weak scaled with a
given level of efficiency. For strong scaling, METG can be used to
compute the scale at which performance can be expected to dip below
that level of efficiency. In either case, METG is a useful, intuitive
measure as it
encapsulates the task granularity at which systems can be relied on to
execute efficiently. As a rule of thumb, as long as an application's
average task granularity exceeds the METG, the application should
execute efficiently.

Together, METG and Task Bench help to mitigate issues which are common
in limit studies on the task scheduling throughput of runtime
systems. Such studies typically employ a metric called \emph{tasks per
  second} (TPS), which is almost universally measured with trivial
dependencies and empty tasks. While TPS is certainly an upper bound on
task scheduling throughput, it is not a tight bound, as the cost of
non-trivial dependencies can be significant and sometimes adversely
impact scalability. This issue is easily fixed by running non-trivial
configurations of Task Bench.

But there is a deeper issue with the use of empty tasks. In all
realistic scenarios, runtime systems must share resources with the
application, either explicitly by e.g., allocating cores for runtime
use, or implicitly by sharing a time slice on the same cores used for
application tasks, and other resources such as memory bandwidth and
processor cache capacity. The use of empty tasks makes the application
share of these resources appear to be zero, which permits usage by the
runtime system to be unconstrained. How then is one supposed to
determine what resource usage to consider reasonable on the part of
the runtime system? Only by requiring the application usage of
resources to meet a certain minimum level, as in METG, can we ensure
that runtime system resource usage is constrained to realistic levels.

We conduct a comprehensive comparative study of Task Bench
implementations in 13 different programming systems, including
Chapel~\cite{Chapel07}, Charm++~\cite{Charmpp93}, MPI~\cite{MPI},
OmpSs~\cite{OmpSs11}, OpenMP~\cite{OpenMPSpec40},
PaRSEC~\cite{PARSEC13}, Realm~\cite{Realm14}, Regent~\cite{Regent15},
Spark~\cite{Spark10}, StarPU~\cite{StarPU11},
Swift/T~\cite{Wozniak13}, TensorFlow~\cite{TensorFlow15}, and
X10~\cite{X1005}. Using the METG metric, we determine the baseline
runtime overhead of each system along with overheads associated with
increasing number and/or variations in the pattern of dependencies. We
also explore each system's ability to scale, hide communication
overhead, and mitigate load imbalance.

The following four sections each describe a contribution of this
paper:

\begin{itemize}
\item Section~\ref{sec:metg} defines the METG metric and its
  relationship to quantities of interest to application developers.
\item Section~\ref{sec:task-bench} describes the design of Task Bench.
\item Section~\ref{sec:implementation} details the implementation of
  Task Bench in 13 programming systems.
\item Section~\ref{sec:evaluation} provides a comprehensive evaluation
  of Task Bench performance on up to 256 Haswell nodes of the Cori
  supercomputer.
\end{itemize}

Section~\ref{sec:related-work} relates this work to previous efforts,
and Section~\ref{sec:conclusion} concludes.
