\section{Introduction}
\label{sec:introduction}

Challenges in heterogeneous and extreme-scale parallel and distributed
computing have lead to the rise of a wide variety of new programming
models and runtime systems. However, efforts to understand the
performance behavior of these system have been limited. In many cases,
programming models are evaluated with weak and/or strong scaling of a
bespoke set of carefully hand-written applications, which make it
difficult to understand how the results generalize to other
applications. Furthermore, these are not limit studies, and thus do
not help to elucide the boundaries of the performance envelops of the
respective systems.

We introduce \emph{minimum effective task granularity} (METG), a
metric of application performance specifically designed to explore the
fundamental performance behavior of parallel and distributed runtime
systems. METG can be used to identify the baseline overhead, limit of
scalability, and ability to hide and overlap communication of such
systems. A key insight of the METG metric is that large \emph{tasks},
or units of work, can be used to hide almost any amount of
overhead. Thus METG considers the \emph{minimum} task granularity that
can be executed. This is a lower bound, so as long as the average task
granularity of a given application exceeds this value, the runtime
system should be able to execute it efficiently.

A variety of runtime systems are able to make use of parallelism
internally for a variety of purposes, from analysis of tasks, to
scheduling or the orchestration of data movement and
synchronization. In order to mitigate unwanted influence due to
unreasonable allocation of resources to such internal uses, METG
concerns \emph{effective} task execution, where the overall
computational efficiency meets some threshold (say, 50\%). If the
METG(50\%) of an application is, say, 1~ms, then this indicates that
the application can be executed with as small as 1~ms tasks while
maintaining at least 50\% overall efficiency.

We present Task Bench, a benchmark designed specifically to study the
performance behavior of parallel and distributed runtime systems at
the limits of runtime overhead. Task Bench is a \emph{parameterized}
benchmark, characterized by an \emph{iteration space} of tasks (with
one task per point in the iteration space) and a \emph{dependence
  relation} which determines how tasks depend on other tasks. By
formulating the benchmark in this way, we make it possible to rapidly
simulate a wide variety of communication patterns relevant in
scientific computing, such as trivial parallelism, halo exchanges (for
structured and unstructured mesh-based computations), sweeps (such as
for the discrete ordinates method), FFTs, trees (for divide and
conquer algorithms), and so on. Task Bench can be configured to
execute a variety of kernels, such as compute- and memory-bound
kernels, and load imbalanced variants of the above. Task Bench can
also be used to stress communication-bound cases, by configuring the
amount of data that is produced by each task.

In order to explore the impact of task parallelism on runtime
performance, Task Bench can also execute multiple task graphs
simultaneously. Runtime systems which support asynchronous execution
can use this additional scheduling flexibility to hide communication
overheads or to mitigate load imbalance. METG makes it possible to
quantify the degree to which various systems are successful at doing
so, and to measure if (and when) such factors are more, or less,
important than baseline runtime overhead.

We provide a comprehensive evaluation of Task Bench implementations in
13 different runtime systems, including Chapel~\cite{Chapel07},
Charm++~\cite{Charmpp93}, MPI~\cite{MPI}, OmpSs~\cite{OmpSs11},
OpenMP~\cite{OpenMPSpec40}, PaRSEC~\cite{PARSEC13},
Realm~\cite{Realm14}, Regent~\cite{Regent15}, Spark~\cite{Spark10},
StarPU~\cite{StarPU11}, Swift/T~\cite{Wozniak13},
TensorFlow~\cite{TensorFlow15}, and X10~\cite{X1005}. Using the METG
metric, we determine the baseline runtime overhead of each system
along with overheads associated with increasing number and/or
variations in the pattern of dependencies. We also explore each
system's ability to scale, hide communication overhead, and mitigate
load imbalance.

The following four sections each describe a contribution of this
paper:

\begin{itemize}
\item Section~\ref{sec:metg} defines the METG metric and its
  relationship to quantities of interest to application developers.
\item Section~\ref{sec:task-bench} describes the design of Task Bench.
\item Section~\ref{sec:implementation} details the implementation of
  Task Bench in 13 runtime systems.
\item Section~\ref{sec:evaluation} provides a comprehensive evaluation
  of Task Bench performance on up to 64 nodes of the Cori
  supercomputer.
\end{itemize}

Section~\ref{sec:related-work} relates this work to previous efforst,
and Section~\ref{sec:conclusion} concludes.
